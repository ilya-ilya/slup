\documentclass[a4paper, 12pt]{report}
\makeindex

\usepackage[utf8]{inputenc}
\usepackage[russian]{babel}
%\usepackage{pscyr}
\usepackage{mathtools}
\usepackage{amsmath, amssymb, amsfonts}
\usepackage{graphicx}
\usepackage[update]{epstopdf}
\usepackage[nosimple,dots,thmitshape,nodiagram]{dmvn}
\usepackage[all]{xy}
\CompileMatrices

%	XXX:К сожалению, я знаю, что это делает, пацаны говорят и это правда,
%	что можно будет кликать по ссылкам, и на практике оказалось, что да.
\usepackage[unicode]{hyperref}
\hypersetup{
	debug = true,
}

\textwidth=500pt
\textheight=750pt
\oddsidemargin=20pt
\hoffset=-1.5cm
\topmargin=-25mm
\tolerance=4000

% from Boris with love
\DeclareMathOperator{\var}{\mathrm var}
\DeclareMathOperator{\Exp}{\mathrm Exp}
\DeclareMathOperator{\Pois}{\mathrm Pois}
\begin{document}
% good reference for equations
\renewcommand{\theequation}{\thesection.\arabic{equation}}
% At least, because ex counter resets at start of section
\renewcommand{\theex}{\thesection.\arabic{ex}}
% better equal by definition
\renewcommand{\eqdef}{\triangleq}

% Transoceanization
\renewcommand{\emptyset}{\varnothing}
\renewcommand{\phi}{\varphi}
\renewcommand{\epsilon}{\varepsilon}

% Львовский называл это dirty tricks,
% но идеалогически единицей курса является именно лекция
\renewcommand{\chaptername}{Лекция}

\newcommand{\Cx}{\mathbb{C}}
\newcommand{\Hx}{\mathbb{H}}
\newcommand{\Zm}[1]{\mathbb{Z}_{#1}}
\newcommand{\fA}{\forall\;}
\newcommand{\Ex}{\exists\;}
\newcommand{\Exo}{\exists\,!\;}
\newcommand{\upto}{\nearrow}
% from Львовский with love
\newcommand*{\diff}[1]{\, d#1}


\dmvntitle{Булинский А.В.}{Случайные процессы}{6 семестр, втрой поток}{}{\today}
%FIXME: переносы формул
\tableofcontents

\documentclass[a4paper, 12pt]{report}
\makeindex

\usepackage[utf8]{inputenc}
\usepackage[russian]{babel}
%\usepackage{pscyr}
\usepackage{mathtools}
\usepackage{amsmath, amssymb, amsfonts}
\usepackage{graphicx}
\usepackage[update]{epstopdf}
\usepackage[nosimple,dots,thmitshape,nodiagram]{dmvn}
\usepackage[all]{xy}
\CompileMatrices

%	XXX:К сожалению, я знаю, что это делает, пацаны говорят и это правда,
%	что можно будет кликать по ссылкам, и на практике оказалось, что да.
\usepackage[unicode]{hyperref}
\hypersetup{
	debug = true,
}

\textwidth=500pt
\textheight=750pt
\oddsidemargin=20pt
\hoffset=-1.5cm
\topmargin=-25mm
\tolerance=4000

% from Boris with love
\DeclareMathOperator{\var}{\mathrm var}
\DeclareMathOperator{\Exp}{\mathrm Exp}
\DeclareMathOperator{\Pois}{\mathrm Pois}
\begin{document}
% good reference for equations
\renewcommand{\theequation}{\thesection.\arabic{equation}}
% At least, because ex counter resets at start of section
\renewcommand{\theex}{\thesection.\arabic{ex}}
% better equal by definition
\renewcommand{\eqdef}{\triangleq}

% Transoceanization
\renewcommand{\emptyset}{\varnothing}
\renewcommand{\phi}{\varphi}
\renewcommand{\epsilon}{\varepsilon}

% Львовский называл это dirty tricks,
% но идеалогически единицей курса является именно лекция
\renewcommand{\chaptername}{Лекция}

\newcommand{\Cx}{\mathbb{C}}
\newcommand{\Hx}{\mathbb{H}}
\newcommand{\Zm}[1]{\mathbb{Z}_{#1}}
\newcommand{\fA}{~\forall\;}
\newcommand{\Ex}{~\exists\;}
\newcommand{\Exo}{~\exists\,!\;}
\newcommand{\Pbb}{\mathbb{P}}


\dmvntitle{Булинский А.В.}{Случайные процессы}{6 семестр, втрой поток}{}{\today}
%FIXME: переносы формул
\tableofcontents

\input{slupy6.Bulinsky.ind}
 % TOC и Предметный указатель
\chapter{Случайные блуждания}

\section{Понятие случайного блуждания}

\begin{df}\index{Измеримое!пространство}
	Пусть $V$ --- множество, а $\As$ --- $\si$-алгебра его подмножеств.
	Тогда $(V, \As)$ называется \textit{измеримым пространством}.
\end{df}

\begin{df}\index{Измеримое!отображение}
 	Пусть есть $(V, \As)$ и $(S, \Bs)$ --- два измеримых пространства, 
	$f \cln V \to S$ --- отображение.
	$f$ называется \textit{$\As \divs \Bs$-измеримым}, если $\fA B \in \Bs f^{-1}(B) \in \As$.
\end{df}
\begin{denote}
	$f\in \As \divs \Bs$.
\end{denote}

\begin{df}\index{Случайный!элемент}
	Пусть есть $(\Om, \Fs, \Pbb)$ --- вероятностное пространство,
	$(S, \Bs)$ --- измеримое пространство,
	$Y \cln \Om \to S$ --- отображение.
	Если $Y \in \Fs \divs \Bs$, то $Y$ называется \textit{случайным элементом}.
\end{df}

\begin{ex}
	$S = \R^m$, $\Bc = \Bc(\R)$ --- борелевские множества.
	Тогда при $m > 1$ случайный элемент $Y$ --- случайный вектор;
	если $m = 1$, то $Y$ --- случайная величина.
	$\Pb_Y(B) = \Pb[Y^{-1}(B)]$ --- мера на $\Bc$.

	Легко видеть, что
	$$
		\Pb_Y(B) = \Pb\hc{\om \in \Om | Y(\om) \in B}
	$$
\end{ex}

\begin{df}\index{Распределение!случайного элемента}
 	Пусть $(\Om, \Fs, \Pbb)$ --- вероятностное пространство,
	$(S, \Bs)$ --- измеримое пространство,
	$Y \cln \Om \to S$ --- случайный элемент.
	\textit{Распределение вероятностей, индуцированное случайным элементом $Y$,}
	--- это функция на множествах из $\Bs$, задаваемая равенством
	$$
		\Pbb_Y (B):= \Prob(Y^{-1}(B)), \quad B\in\mathscr{B}.
	$$
\end{df}

\begin{df}\index{Случайный!процесс}
	Пусть $(S_t, \Bs_t)_{t \in T}$ --- семейство измеримых пространств.
	\textit{Случайный процесс, ассоциированный с этим семейством,} --- это семейство случайных элементов
	$X = \hc{X(t) | t \in T}$, где $X(t) \cln \Om \to S_t$,
	$X(t) \in \Fs \divs \Bs_t \fA t \in T$.
	Здесь $T$ --- это произвольное параметрическое множество,
	$(S_t, \Bs_t)$ --- произвольные измеримые пространства.
\end{df}

\begin{note}
	Если $T \subs \R$, то $t \in T$ интерпретируется как время.
	Если $T = \R$, то время \textit{непрерывно};
	если $T = \Z$ или $T = \Z_+$, то время \textit{дискретно};
	если $T \subs \R^d$, то говорят о \textit{случайном поле}.
\end{note}

\begin{df}
	Случайные элементы $X_1 \sco X_n$ называются \textit{независимыми}, если
	$$
		\Pbb \hr{\capkun \hc{X_k \in  B_k}} = \prodl{k=1}{n} \Pbb(X_k \in B_k),
			\quad \fA B_1 \in \Bs_1 \sco B_n \in \Bs_n.
	$$
\end{df}

\begin{theorem}[Ломницкого-Улама]\index{Теорема!Ломницкого-Улама}
 	Пусть $(S_t, \Bs_t, \Q_t)_{t \in  T}$ --- семейство вероятностных пространств.
	Тогда на некотором $(\Om, \Fs, \Pbb)$ существует семейство \textit{независимых} случайных элементов
	$X_t \cln \Om \to S_t$, $X_t \in \Fs \divs \Bs_t$ таких, что $\Pbb_{X_t} = \Q_{t}, t \in T$.
\end{theorem}

\begin{note}
	Это значит, что на некотором вероятностном пространстве можно
	задать независимое семейство случайных элементов с наперед указанными распределениеми.
	При этом $T$ по-прежнему любое, как и $(S_t, \Bs_t, \Q)_{t \in T}$ --- произвольные вероятностные пространства.
	Независимость здесь означает независимость в совокупности для любого конечного поднабора.
\end{note}

\section{Случайные блуждания}

\begin{df}\index{Случайное блуждание}
  Пусть $X$, $X_{1}$, $X_{2}$, \ldots - независимые одинаково распределенные случайные векторы со значениями в $\mathbb{R}^{d}$. \emph{Случайным блужданием в $\mathbb{R}^{d}$} называется случайный процесс с дискретным временем $S = \lbrace S_{n}, n \geqslant 0 \rbrace$ ($n \in \mathbb{Z}_{+}$) такой, что
  \begin{align*}
    S_{0} &:= x \in \mathbb{R}^{d} \quad\text{(начальная точка)};\\
    S_{n} &:= x + X_{1} + \ldots + X_{n}, \quad n \in \mathbb{N}.
  \end{align*}
\end{df}

\begin{df}\index{Случайное блуждание!простое}
  \emph{Простое случайное блуждание в $\mathbb{Z}^{d}$} "--- это такое случайное блуждание, что
  \begin{equation*}
    \Prob(X = e_{k}) = \Prob(X = -e_{k}) = \frac{1}{2d},
  \end{equation*}
  где $e_{k} = (0, \ldots, 0, \underbrace{1}_{k}, 0, \ldots, 0)$, $k = 1, \ldots, d$.
\end{df}

\begin{df}\index{Случайное блуждание!простое!возвратное}
  Введем N := $\sum\limits_{n=0}^\infty \ind \lbrace S_{n} = 0 \rbrace$ ($\leqslant \infty$). Это, по сути, число попаданий нашего процесса в точку 0. Простое случайное блуждание $S = \lbrace S_{n}, n \geqslant 0\rbrace$ называется \emph{возвратным}, если $\Prob(N = \infty) = 1$; \emph{невозвратным}, если $\Prob(N < \infty) = 1$.
\end{df}

\begin{note}
  Следует понимать, что хотя определение подразумевает, что $\Prob(N = \infty)$ равно либо 0, либо 1, пока что это является недоказанным фактом. Это свойство будет следовать из следующей леммы.
\end{note}

\begin{note}[от наборщика]
  Судя по всему, в лемме ниже подразумевается, что начальная точка нашего случайного блуждания "--- это 0.
\end{note}
\begin{df}
  Число $\tau := \inf\lbrace n \in \mathbb{N} : S_{n} = 0 \rbrace$ ($\tau := \infty$, если $S_{n} \neq 0$ $\forall\, n \in N$) называется \emph{моментом первого возвращения в 0}.
\end{df}

\begin{lem}
  Для  $ \forall\, n \in \mathbb{N} \; \Prob(N = n)  =  \Prob(\tau = \infty)\Prob(\tau < \infty)^{n-1}$.
\end{lem}

\begin{proof}
  При $n = 1$ формула верна: $\lbrace N = 1 \rbrace = \lbrace \tau = \infty \rbrace$. Докажем по индукции.

  \begin{multline*}
    \Prob(N = n+1, \tau < \infty) = \sum_{k=1}^{\infty} \Prob(N = n+1, \tau = k) =\\
    = \sum_{k=1}^{\infty} \Prob\left( \sum_{m=0}^{\infty} \ind \lbrace S_{m+k} - S_{k} = 0 \rbrace = n, \tau = k\right) =\\
    = \sum_{k=1}^{\infty} \Prob\left( \sum_{m=0}^{\infty} \ind \left\lbrace S_{m} = 0 \right\rbrace = n\right)\Prob(\tau = k) =\\
    = \sum_{k=1}^{\infty} \Prob(N^{\prime} = n)\Prob(\tau = k),
  \end{multline*}
  где $N^{\prime}$ определяется по последовательности $X_{1}^{\prime} = X_{k+1}$, $X_{2}^{\prime} = X_{k+2}$ и так далее. Из того, что $X_{i}$ --- независиые одинаково распределенные случайные векторы, следует, что $N^{\prime}$ и $N$ распределены одинаково. Таким образом, получаем, что
  \begin{equation*}
    \Prob(N = n+1, \tau < \infty) = \Prob(N = n)\Prob(\tau < \infty).
  \end{equation*}
  Заметим теперь, что
  \begin{equation*}
    \Prob(N = n+1) = \Prob(N = n+1, \tau < \infty) + \Prob(N = n+1, \tau = \infty),
  \end{equation*}
  где второе слагаемое обнуляется из-за того, что $n+1 \geqslant 2$. Из этого следует, что
  \begin{equation*}
    \Prob(N = n+1) = \Prob(N = n)\Prob(\tau < \infty).
  \end{equation*}
  Пользуемся предположением индукции и получаем, что
  \begin{equation*}
    \Prob(N = n+1) = \Prob(\tau = \infty)\Prob(\tau < \infty)^{n},
  \end{equation*}
  что и завершает доказательство леммы.
\end{proof}

\begin{cor}
  $\Prob(N = \infty)$ равно 0 или 1. $\Prob(N < \infty) = 1 \Leftrightarrow \Prob(\tau < \infty) < 1$.
\end{cor}

\begin{proof}
  Пусть $\Prob(\tau < \infty) < 1$. Тогда
  \begin{flushleft}
    $\Prob(N < \infty) = \sum\limits_{n=1}^{\infty} \Prob(N = n) = \sum\limits_{n=1}^{\infty} \Prob(\tau = \infty) \Prob(\tau < \infty)^{n-1} = \frac{\Prob(\tau = \infty)}{1 - \Prob(\tau < \infty)} = \frac{\Prob(\tau = \infty)}{\Prob(\tau = \infty)} = 1$.
  \end{flushleft}
  Это доказывает первое утверждение следствия и импликацию справа налево в формулировке следствия. Докажем импликацию слева направо.
  \begin{flushleft}
    $\Prob(\tau < \infty) = 1 \Rightarrow \Prob \left((\tau = \infty ) = 0 \right) \Rightarrow \Prob(N = n) = 0$ $\forall\, n \in \mathbb{N} \Rightarrow \Prob(N < \infty) = 0$.
  \end{flushleft}
  Следствие доказано.
\end{proof}

\begin{thm}
  Простое случайное блуждание в $\mathbb{Z}^{d}$ возвратно $\Leftrightarrow$ $\Expect N = \infty$ (соответственно, невозвратно $\Leftrightarrow$ $\Expect N < \infty$).
\end{thm}

\begin{proof}
  Если $\Expect N < \infty$, то $\Prob(N<\infty) = 1$.
  Пусть теперь $\Prob(N<\infty) = 1$. Это равносильно тому, что $\Prob(\tau < \infty) < 1$.
  \begin{multline*}
    \Expect N = \sum_{n=1}^{\infty} n\Prob(N=n) = \sum\limits_{n=1}^{\infty} n\Prob(\tau = \infty)\Prob(\tau < \infty)^{n-1} = \\ =\Prob(\tau = \infty)\sum\limits_{n=1}^{\infty} n\Prob(\tau < \infty)^{n-1}.
  \end{multline*}
  Заметим, что
  \begin{equation*}
    \sum_{n=1}^{\infty} np^{n-1} = (\sum\limits_{n=1}^{\infty} p^{n})^{\prime} = (\frac{1}{1-p})^{\prime} = \frac{1}{(1-p)^{2}}.
  \end{equation*}
  Тогда, продолжая цепочку равенств, получаем, что
  \begin{equation*}
    \Prob(\tau = \infty)\sum_{n=1}^{\infty} n\Prob(\tau < \infty)^{n-1} = \frac{\Prob(\tau = \infty)}{(1 - \Prob(\tau < \infty))^{2}} = \frac{1}{1 - \Prob(\tau < \infty)},
  \end{equation*}
  что завершает доказательство теоремы.
\end{proof}

\begin{note}
  Заметим, что поскольку $N = \sum\limits_{n=0}^{\infty} \ind \lbrace S_{n} = 0 \rbrace$, то
  \begin{equation*}
    \Expect N = \sum_{n=0}^{\infty} \Expect \ind \lbrace S_{n} = 0 \rbrace = \sum\limits_{n=0}^{\infty} \Prob(S_{n} = 0),
  \end{equation*}
  где перестановка местами знаков матожидания и суммы возможна в силу неотрицательности членов ряда. Таким образом, \begin{center}
    S возвратно $\Leftrightarrow$ $\sum\limits_{n=0}^{\infty} \Prob(S_{n} = 0) = \infty$.
  \end{center}
\end{note}

\begin{cor}
  $S$ возвратно при $d = 1$ и $d = 2$.
\end{cor}

\begin{proof}
  $\Prob(S_{2n} = 0) = (\frac{1}{2d})^{2n} \sum_{\substack{n_{1}, \ldots, n_{d} \geqslant 0 \\ n_{1} + \ldots + n_{d} = n}} \frac{(2n)!}{(n_{1}!)^{2} \ldots (n_{d}!)^{2}}$.
  \begin{flushleft}
    \emph{Случай d = 1}: $\Prob(S_{2n} = 0) = \frac{(2n)!}{(n!)^{2}}(\frac{1}{2})^{2n}$.
  \end{flushleft}Согласно формуле Стирлинга,
  \begin{equation*}
    m! \sim \left(\frac{m}{e}\right)^{m} \sqrt{2 \pi m}, \quad m \rightarrow \infty.
  \end{equation*}
  Соответственно,
  \begin{equation*}
    \Prob(S_{2n} = 0) \sim \frac{1}{\sqrt{\pi n}} \Rightarrow
  \end{equation*}
  $\Rightarrow$ ряд $\sum\limits_{n=0}^{\infty} \frac{1}{\sqrt{\pi n}} = \infty \Rightarrow$ блуждание возвратно.
  Аналогично рассматривается \emph{случай d = 2}: $\Prob(S_{2n} = 0) = \ldots = \left\lbrace \frac{(2n)!}{(n!)^{2}}(\frac{1}{2})^{2n} \right\rbrace ^{2}$ $\sim \frac{1}{\pi n} \Rightarrow$ ряд тоже разойдется $\Rightarrow$ блуждание возвратно. Теорема доказана.
\end{proof}

\subsection{Исследование случайного блуждания с помощью характеристической функции}

\begin{thm}
  Для простого случайного блуждания в $\mathbb{Z}^{d}$
  \begin{equation*}
    \Expect N = \lim\limits_{c \uparrow 1} \frac{1}{(2 \pi)^{d}} \int\limits_{[-\pi, \pi]^{d}} \frac{1}{1 - c \varphi (t)}\,\dif t,
  \end{equation*}
  где $\varphi (t)$ "--- характеристическая функция X, $t \in \mathbb{R}^{d}$.
\end{thm}

\begin{proof}
  $\int\limits_{[-\pi, \pi]} \frac{e^{inx}}{2 \pi} dx = \begin{cases}
    1, & n=0 \\ 0, &n \neq 0
  \end{cases}$. Следовательно,
  \begin{equation*}
    \ind \lbrace S_{n} = 0 \rbrace  = \prod_{k=1}^{d} \ind \lbrace S_{n}^{(k)} = 0 \rbrace = \prod\limits_{k=1}^{d} \int\limits_{[-\pi, \pi]} \frac{e^{i S_{n}^{(k)} t_{k}}}{2 \pi}\,\dif t_{k} = \frac{1}{(2 \pi)^{d}} \int\limits_{[-\pi, \pi]^{d}} e^{i (S_{n}, t)}\,\dif t{.}
  \end{equation*}
  По теореме Фубини
  \begin{equation*}
    \Expect \ind (S_{n} = 0) = \Expect \frac{1}{(2 \pi)^{d}} \int_{[-\pi, \pi]^{d}} e^{i (S_{n}, t)}\,\dif t = \frac{1}{(2 \pi)^{d}} \int\limits_{[-\pi, \pi]^{d}} \Expect e^{i (S_{n}, t)}\,\dif t.
  \end{equation*}
  Заметим, что
  \begin{equation*}
    \Expect e^{i (S_{n}, t)} = \prod_{k=1}^{n} \varphi_{X_{k}} (t) = (\varphi (t))^{n}.
  \end{equation*}
  Тогда
  \begin{equation*}
    \Expect \ind (S_{n} = 0) = \Prob(S_{n} = 0) = \frac{1}{(2 \pi)^{d}}\int_{[-\pi, \pi]^{d}} \left(\varphi \left(t\right)\right)^{n}\,\dif t.
  \end{equation*}
  Из этого следует, что
  \begin{equation*}
    \sum_{n=0}^{\infty} c^{n} \Prob(S_{n} = 0) = \frac{1}{(2 \pi)^{d}} \int\limits_{[-\pi, \pi]^{d}} \sum\limits_{n=0}^{\infty} (c \varphi(t))^{n}\,\dif t{,}\quad\text{где $0 < c < 1$}.
  \end{equation*}
  Поскольку $|c \varphi| \leqslant c < 1$, то
  \begin{equation*}
    \frac{1}{(2 \pi)^{d}} \int_{[-\pi, \pi]^{d}} \sum\limits_{n=0}^{\infty} (c \varphi(t))^{n}\,\dif t = \frac{1}{(2 \pi)^{d}} \int\limits_{[-\pi, \pi]^{d}} \frac{1}{1 - c \varphi (t)}\,\dif t
  \end{equation*}
  по формуле для суммы бесконечно убывающей геометрической прогрессии. Осталось только заметить, что
  \begin{equation*}
    \sum_{n=0}^{\infty} c^{n} \Prob(S_{n} = 0) \rightarrow \sum\limits_{n=0}^{\infty} \Prob(S_{n} = 0) = \Expect N, \quad c \uparrow 1,
  \end{equation*}
  что и завершает доказательство теоремы.
\end{proof}

\begin{cor}
  При $d \geqslant 3$ простое случайное блуждание невозвратно.
\end{cor}

\begin{note}
  \sloppy
  Можно говорить и о случайных блужданиях в $\mathbb{R}^d$, если $X_{i}: \Omega \rightarrow \mathbb{R}^d$. Но тогда о возвратности приходится говорить в терминах бесконечно частого попадания в $\varepsilon$-окрестность точки $x$.
\end{note}

\begin{df}\index{Множество!возвратности}
  Пусть есть случайное блуждание $S$ на $\mathbb{R}^d$. Тогда \emph{множество возвратности} случайного блуждания $S$ "--- это множество
  \begin{equation*}
    R(S) = \lbrace x \in \mathbb{R}^d : \text{блуждание возвратно в окрестности точки } x \rbrace \text{.}
  \end{equation*}
\end{df}

\begin{df}\index{Множество!достижимости}
  Пусть есть случайное блуждание $S$ на $\mathbb{R}^d$. Тогда \emph{точки, достижимые случайным блужданием $S$,} "--- это множество $P(S)$ такое, что
  \begin{equation*}
    \forall\, z \in P(S) \; \; \forall\, \varepsilon > 0 \; \; \exists\, n \negmedspace : \; \, \Prob( \| S_{n} - z \| < \varepsilon) > 0 \text{.}
  \end{equation*}
\end{df}

\begin{thm}[Чжуна-Фукса]\index{Теорема!Чжуна-Фукса}
  Если $R(S) \neq \varnothing$, то $R(S) = P(S)$.
\end{thm}

\begin{cor}
  Если $0 \in R(S)$, то $R(S) = P(S)$; если
  $0 \notin R(S)$, то  $R(S) = \varnothing$.
\end{cor}

\section[Лекция от 15.02.17. Ветвящиеся процессы и процессы восстановления]{Лекция от 15.02.17\\ {\large Ветвящиеся процессы и процессы восстановления}}

\subsection{Модель Гальтона--Ватсона}\index{Модель Гальтона-Ватсона}

\paragraph{Описание модели}

Пусть $\lbrace \xi{,} \, \xi_{n, k}{,}\: n, k \in \mathbb{N}\rbrace$ "--- массив независимых одинаково распределенных случайных величин,
\begin{equation*}
  \Prob (\xi = m) = p_{m} \geqslant 0,\; \; m \in \mathbb{Z}_{+} = \lbrace 0, 1, 2, \ldots \rbrace.
\end{equation*}
Такие существуют в силу теоремы Ломницкого--Улама. Положим
\begin{equation*}
  \begin{aligned}
    Z_{0}(\omega) &:= 1,\\
    Z_{n}(\omega) &:= \sum_{k=1}^{Z_{n-1}(\omega)} \xi_{n, k}(\omega) \quad \text{для $n \in \nat$}.
  \end{aligned}
\end{equation*}
Здесь подразумевается, что если $Z_{n-1}(\omega) = 0$, то и вся сумма равна нулю.
Таким образом, рассматривается сумма случайного числа случайных величин. Определим
$A = \lbrace \omega\colon \exists\, n = n(\omega)\; Z_{n}(\omega) = 0 \rbrace$ "--- \index{Вырождение}\emph{событие вырождения популяции}.
Заметим, что если $Z_{n}(\omega) = 0$, то $Z_{n+1}(\omega) = 0$. Таким образом,
$\lbrace Z_{n} = 0 \rbrace \subset \lbrace Z_{n+1} = 0 \rbrace$ и $A = \bigcup\limits_{n=1}^{\infty} \lbrace Z_{n} = 0 \rbrace.$

По свойству непрерывности вероятностной меры,
\begin{equation*}
  \Prob(A) = \lim_{n \to \infty} \Prob(Z_{n} = 0).
\end{equation*}

\begin{df}\index{Производящая функция}
  Пусть дана последовательность $(a_{n})_{n=0}^{\infty}$ неотрицательных чисел такая, что $\sum\limits_{n=0}^{\infty} a_{n} = 1$.
  \emph{Производящая функция} для этой последовательности "--- это
  \begin{equation*}
    f(s) := \sum_{k=0}^{\infty} s^{k}a_{k} {,}\quad |s| \leqslant 1
  \end{equation*}
  (нас в основном будут интересовать $s \in [0, 1]$).
\end{df}

Заметим, что если $a_{k} = \Prob(Y = k)$, $k = 0, 1, \ldots$ , то
\begin{equation*}
  f_{Y}(s) = \sum_{k=0}^{\infty} s^{k} \Prob(Y = k) = \Expect s^{Y} \! \! {,} \quad s \in [0, 1]{.}
\end{equation*}

\begin{lem}
  \label{lem1}
  Вероятность $\Prob(A)$ является корнем уравнения $\psi(p) = p$, где $\psi = f_{\xi}$ и $p \in [0, 1]$.
\end{lem}

\begin{proof}

  \begin{multline*}
    f_{Z_{n}}(s) = \Expect s^{Z_{n}} = \Expect \left(s^{\sum_{k=1}^{Z_{n-1}} \xi_{n, k}}\right) =\\
    = \sum_{j=0}^{\infty} \Expect \left[ \left( s^{\sum_{k=1}^{Z_{n-1}} \xi_{n, k}} \right)  \ind \lbrace Z_{n-1} = j \rbrace \right] = \\ = \sum_{j=0}^{\infty} \Expect \left[ \left( s \: ^{\sum_{k=1}^{j} \xi_{n, k}} \right)  \ind \lbrace Z_{n-1} = j \rbrace \right].
  \end{multline*}
  Поскольку $\sigma \lbrace Z_{r} \rbrace \subset \sigma \lbrace \xi_{m, k}, \; m = 1, \ldots, r, \; k \in \mathbb{N} \rbrace$, которая независима с $\sigma \lbrace \xi_{n, k}, \; k \in \mathbb{N} \rbrace$ (строгое и полное обоснование остается в качестве упражнения), то
  \begin{multline*}
    \sum_{j=0}^{\infty} \Expect \left[ \left( s \: ^{\sum_{k=1}^{j} \xi_{n, k}} \right)  \ind \lbrace Z_{n-1} = j \rbrace \right]  =  \sum\limits_{j=0}^{\infty} \Expect \left(s \: ^{\sum\limits_{k=1}^{j} \xi_{n, k}}\right) \Expect \ind \lbrace Z_{n-1} = j \rbrace  = \\ = \sum\limits_{j=0}^{\infty} \Expect \left(s \: ^{\sum\limits_{k=1}^{j} \xi_{n, k}}\right) \Prob ( Z_{n-1} = j ) = \sum\limits_{j=0}^{\infty} \prod\limits_{k=1}^{j} \Expect s^{\xi_{n, k}} \Prob (Z_{n-1} = j) = \\ = \sum\limits_{j=0}^{\infty} \psi_{\xi}^{j} (s) \Prob (Z_{n-1} = j)  =  f_{Z_{n-1}} \left(\psi_{\xi} \left(s \right) \right)
  \end{multline*}
  в силу независимости и одинаковой распределенности $\xi_{n, k}$ и определения производящей функции. Таким образом,
  \begin{equation*}
    f_{Z_{n}} (s) = f_{Z_{n-1}} \left( \psi_{\xi} \left(s \right)\right){,}\quad s \in [0, 1]{.}
  \end{equation*}
  Подставим $s = 0$ и получим, что
  \begin{equation*}
    f_{Z_{n}} (0) = f_{Z_{n-1}} \left( \psi_{\xi} \left(0 \right)\right)
  \end{equation*}
  Заметим, что
  \begin{multline*}
    f_{Z_{n}}(s) = f_{Z_{n-1}}(\psi_{\xi}(s)) = f_{Z_{n-2}} \left(\psi_{\xi} \left( \psi_{\xi} \left(s \right) \right) \right) = \ldots = \underbrace{\psi_{\xi} (\psi_{\xi} \ldots (\psi_{\xi}}_{\text{$n$ итераций}}(s)) \ldots ) = \\ = \psi_{\xi} (f_{Z_{n-1}} (s)){.}
  \end{multline*}
  Тогда при $s = 0$ имеем, что
  \begin{equation*}
    \Prob (Z_{n} = 0) = \psi_{\xi} \left( \Prob\left(Z_{n-1}=0 \right) \right) {.}
  \end{equation*}
  Но $\Prob(Z_{n} = 0) \nearrow \Prob(A)$ при $n \to \infty$ и $\psi_{\xi}$ непрерывна на $[0, 1]$.
  Переходим к пределу при $n \to \infty$. Тогда
  \begin{equation*}
    \Prob(A) = \psi_{\xi} (\Prob(A)){,}
  \end{equation*}
  то есть $\Prob(A)$ "--- корень уравнения $p = \psi_{\xi}(p)$, $p \in [0, 1]$.
\end{proof}

\begin{thm}
  Вероятность $p$ вырождения процесса Гальтона--Ватсона есть \textbf{наименьший} корень уравнения
  \begin{equation}
    \label{eq1}
    \psi(p) = p, \quad p \in [0, 1]{,}
  \end{equation}
  где $\psi = \psi_{\xi}$.
\end{thm}

\begin{proof}
  Пусть $p_{0} := \Prob(\xi = 0) = 0$. Тогда
  \begin{equation*}
    \Prob(\xi \geqslant 1) = 1{,}\quad \Prob\left(\bigcap_{n,k} \left\lbrace \xi_{n,k} \geqslant 1 \right\rbrace \right) = 1{.}
  \end{equation*}
  Поэтому $Z_{n} \geqslant 1$ при $\forall\, n$, то есть $\Prob(A)$ "--- наименьший корень уравнения~\eqref{eq1}.
  Пусть теперь $p_{0} = 1$. Тогда $\Prob(\xi = 0)=1 \Rightarrow \Prob(A)$ "--- наименьший корень уравнения~\eqref{eq1}.
  Пусть, наконец, $0 < p_{0} < 1$. Из этого следует, что $\exists\, m~\in~\mathbb{N}{:}\;\, p_{m} > 0$, а значит, $\psi$ строго возрастает на $[0, 1]$. Рассмотрим
  \begin{equation*}
    \Delta_{n} = \big[\psi_{n}(0),\: \psi_{n+1}\left(0\right)\big){,}\; n = 0, 1, 2, \ldots \; {,} %]
  \end{equation*}
  где $\psi_{n}(s)$ "--- это производящая функция $Z_{n}$. Пусть $s \in \Delta_{n}$. Тогда из монотонности $\psi$ на $[0, 1]$ получаем, что
  \begin{equation*}
    \psi(s) - s \; > \; \psi(\psi_{n}(0)) - \psi_{n+1}(0) \; = \; \psi_{n+1}(0) - \psi_{n+1}(0) \; = \; 0{,}
  \end{equation*}
  что означает, что у уравнения~\eqref{eq1} нет корней на $\Delta_{n} \; \forall\, n \in \mathbb{Z_{+}}$.
  Заметим, что
  \begin{equation*}
    \bigcup_{n=0}^{\infty} \Delta_{n} = \big[0,\: \Prob(A)\big), \; \; \psi_{n}(0) \nearrow \Prob(A){.} %]
  \end{equation*}
  По лемме \ref{lem1} $\Prob(A)$ является корнем уравнения \eqref{eq1}. Следовательно, показано, что $\Prob(A)$ "--- наименьший корень, что и требовалось доказать.
\end{proof}

\begin{thm} \mbox{}
 \begin{enumerate}
   \item\label{firth} Вероятность вырождения $\Prob(A)$ есть нуль $\Longleftrightarrow$ $p_{0} = 0$.
   \item\label{secth} Пусть $p_{0} > 0$. Тогда при $\Expect \xi \leqslant 1$ имеем $\Prob (A) = 1$, при $\Expect \xi > 1$ имеем $\Prob(A) < 1$.
 \end{enumerate}
\end{thm}

\begin{proof}
  Докажем \ref{firth}. Пусть $\Prob(A) = 0$. Тогда $p_{0} = 0$, потому что иначе была бы ненулевая вероятность вымирания $\Prob(A) > \Prob(Z_{1} = 0) = p_{0}$. В другую сторону, если $p_{0} = 0$, то вымирания не происходит (почти наверное) из-за того, что у каждой частицы есть как минимум один потомок (почти наверное).

  Докажем \ref{secth}. Пусть $\mu = \Expect\xi \leqslant 1$. Покажем, что в таком случае у уравнения \eqref{eq1} будет единственный корень, равный $1$.
      \[
         \psi_{\xi}^{\prime} (z) = \sum\limits_{k=1}^{\infty} kz^{k-1}\Prob(\xi = k) \;\; \Rightarrow \;\; \psi_{\xi}^{\prime} (z) > 0 \text{ при } z > 0{,}
      \]
      если только $\xi$ не тождественно равна нулю (в противном случае утверждение теоремы выполнено). Заметим также, что $\psi_{\xi}^{\prime} (z)$ возрастает на $z > 0$. Воспользуемся формулой Лагранжа:
      \[
         1 - \psi_{\xi} (z) = \psi_{\xi} (1) - \psi_{\xi} (z) = \psi_{\xi}^{\prime} (\theta) (1 - z) <  \psi_{\xi}^{\prime} (1) (1-z) \leqslant 1-z {,}
      \]
где $z \in (0, 1)$, в силу монотонности $\psi_{\xi}^{\prime} (z)$. Следовательно, если $z < 1$, то
      \[
         1 - \psi_{\xi}(z) < 1 - z{,}
      \]
      то есть $z=1$ "--- это единственный корень уравнения \eqref{eq1}. Значит, $P(A) = 1$.

Пусть $\mu = \Expect\xi > 1$. Покажем, что в таком случае у уравнения \eqref{eq1} есть два корня, один из которых строго меньше единицы.
      \[
         \psi_{\xi}^{\prime\prime}(z) = \sum\limits_{k=2}^{\infty} k(k-1)z^{k-2}\Prob(\xi = k){,}
      \]
следовательно, $\psi_{\xi}^{\prime\prime}(z)$ монотонно возрастает и больше нуля при $z > 0$. Из этого следует, что $1 - \psi_{\xi}^{\prime} (z)$ строго убывает, причем
 \begin{align*}
   &1 - \psi_{\xi}^{\prime} (0) = 1 - \Prob(\xi = 1) > 0 {,} \\
   &1 - \psi_{\xi}^{\prime} (1) = 1 - \mu < 0 {.}
 \end{align*}
 Рассмотрим теперь $z - \psi_{\xi} (z)$ при $z = 0$. Поскольку  $1 - \psi_{\xi} (1) = 0$, производная этой функции монотонно убывает, а $0 - \psi_{\xi} (0) = -\Prob(\xi = 0) < 0$, то график функции $z - \psi_{\xi} (z)$ пересечет ось абсцисс в двух точках, одна из которых будет лежать в интервале $(0, 1)$. Так как вероятность вырождения $\Prob(A)$ равна наименьшему корню уравнения \eqref{eq1}, то $\Prob(A) < 1$, что и требовалось доказать.
\end{proof}

\begin{cor}
  Пусть $\Expect \xi < \infty$. Тогда $\Expect Z_{n} = (\Expect \xi)^{n},\; n \in \mathbb{N}{.}$
\end{cor}

\begin{proof}
  Доказательство проводится по индукции.

  База индукции: $n=1 \Rightarrow \Expect Z_{1} = \Expect \xi$.

  Индуктивный переход:
  \begin{equation*}
    \Expect Z_{n} = \Expect \left( \sum_{k=1}^{Z_{n-1}} \xi_{n,k} \right) = \sum\limits_{j=0}^{\infty} j \, \Expect \xi \Prob (Z_{n-1} = j) = \Expect \xi \, \Expect Z_{n-1} = \left(\Expect \xi \right)^{n}.
  \end{equation*}
\end{proof}

\begin{df}\mbox{}

  При $\Expect \xi < 1$ процесс называется \emph{докритическим.}

  При $\Expect \xi = 1$ процесс называется \emph{критическим.}

  При $\Expect \xi > 1$ процесс называется \emph{надкритическим.}
\end{df}

\subsection{Процессы восстановления}

\begin{df}\index{Процесс!восстановления}
  Пусть $S_{n} = X_{1} + \ldots + X_{n}$, $n \in \mathbb{N}$, $X, X_{1}, X_{2}, \ldots$ "--- независимые одинаково распределенные случайные величины, $X \geqslant 0$. Положим
  \begin{align*}
    Z(0) &:= 0;\\
    Z(t) &:= \sup \lbrace n \in \mathbb{N}:\; S_{n} \leqslant t \rbrace{,}\quad t > 0.
  \end{align*}
  (здесь считаем, что $\sup \varnothing := \infty$). Таким образом,
  \begin{equation*}
    Z(t, \omega) = \sup \left\lbrace n \in \mathbb{N}: \; S_{n}(\omega) \leqslant t \right\rbrace{.}
  \end{equation*}
  Иными словами,
  \begin{equation*}
    \lbrace Z(t) \geqslant n \rbrace = \lbrace S_{n} \leqslant t \rbrace{.}
  \end{equation*}
  Так определенный процесс $Z(t)$ называется \emph{процессом восстановления}.
\end{df}

\begin{note}
  Полезно заметить, что
  \begin{equation*}
    Z(t) = \sum_{n=1}^{\infty} \ind \lbrace S_{n} \leqslant t \rbrace{,}\;\; t > 0{.}
  \end{equation*}
\end{note}

\begin{df}\label{dfstar}
  Рассмотрим \emph{процесс восстановления} $\{ Z^\star(t),\; t \geqslant 0 \}$, который строится по $Y, Y_{1}, Y_{2}, \ldots$ "--- независимым одинаково распределенным случайным величинам, где $\Prob(Y = \alpha) = p \in (0, 1)$, $\Prob(Y = 0) =  1-p$. Исключаем из рассмотрения случай, когда $Y = C = \const$: если $C = 0$, то $Z(t) = \infty \;\: \forall\, t > 0$; если же $C > 0$, то $Z(t) = \left[\frac{t}{c}\right]$.
\end{df}

\begin{lem}
  Для $l = 0, 1, 2,\ldots$
  \begin{equation*}
    \Prob(Z^{\star}(t) = m) =
    \begin{cases}
      C_{m}^{j} \, p^{j+1} q^{m-j} {,}\; \text{где } j = \left[\frac{t}{\alpha} \right],\; &\text{если } m \geqslant j{;} \\
      0,\; &\text{если } m < j{.}
    \end{cases}
  \end{equation*}
\end{lem}




\section{Лекция от 22.02.17. Пуассоновские процессы}


\subsection{Процессы восстановления (продолжение)}

\begin{df}
  Будем говорить, что дискретная случайная величина $U$ имеет \index{Распределение!геометрическое}\emph{геометрическое распределение} с параметром $p \in (0, 1)$, если для $k = 0, 1, 2, \ldots$ $\Prob(U = k) = (1-p)^{k}p$.
\end{df}

\begin{lem}\label{lemsum}
  \sloppy
  Рассмотрим независимые геометрические величины $U_{0}, \ldots , U_{j}$ с параметром $p \in (0, 1)$, где $j = \left[ \frac{t}{\alpha} \right]$. Тогда
  \begin{equation*}
    \Prob (j + U_{0} + \ldots + U_{j} = m) = \Prob (Z^{\star}(t) = m ).
  \end{equation*}
\end{lem}

\begin{proof}
   Обозначим $M = \left\{ (k_0, \ldots, k_j)\colon k_j \in \nonneg, \sum\limits_{i = 0}^j k_j = m - j\right\}$.
  \begin{multline*}
    \Prob\left( U_0 + \ldots + U_j = m - j\right) = \sum_{(k_0, \ldots, k_j) \in M} \Prob(U_0 = k_0, \ldots, U_j = k_j) =\\
    = \sum_{(k_0, \ldots, k_j) \in M} \Prob(U_0 = k_0)\ldots \Prob(U_j = k_j) = \sum_{(k_0, \ldots, k_j) \in M} p (1 - p)^{k_0}\ldots p (1 - p)^{k_j} =\\
    = \sum_{(k_0, \ldots, k_j) \in M} = p^{j + 1} (1 - p)^{k_0 + \ldots + k_j} =\\
    = p^{j + 1} (1 - p)^{m - j} \#M = C_m^j p^{j + 1} (1 - p)^{m - j}.
  \end{multline*}
\end{proof}

\subsection{Сопоставление исходного процесса восстановления со вспомогательным}

\begin{lem}\label{est}
  Пусть $t \geqslant \alpha$. Тогда $\Expect Z^\star(t) \leqslant At$ и $\Expect Z^\star(t)^2 \leqslant B t^2$, где $A = A(p, \alpha) > 0$, $B = B(p, \alpha) > 0$.
\end{lem}

\begin{proof}
  По лемме \ref{lemsum} $\Expect Z^\star(t) = \Expect(j + U_0 + \ldots + U_j) = j + (j + 1) \Expect U$, где $\Expect U =: a(p) < \infty$ "--- математическое ожидание геометрического распределения.

  Тогда
  \begin{multline*}
    \Expect Z^\star(t) = j + (j + 1) a(p) \leqslant (j + 1) \big(a(p) + 1\big) \leqslant\\
    \leqslant \frac{t + \alpha}{\alpha} \big(a(p) + 1\big) \leqslant \frac{2 t}{\alpha} \big( a(p) + 1 \big) = A t,
  \end{multline*}
  где $A := \frac{2 (a(p) + 1)}{\alpha}$.

  Далее,
  \begin{multline*}
    \Expect Z^\star(t)^2 = \var Z^\star(t) + \big(\Expect Z^\star(t) \big)^2 \leqslant (j + 1) \underbrace{\var U}_{\sigma^2(p)} + (j + 1)^2 \big( a(p) + 1 \big)^2 \leqslant\\
    \leqslant (j + 1)^2 \left( \sigma^2(p) + \big( a(p) + 1 \big)^2 \right) \leqslant \frac{4}{\alpha^2} \left( \sigma^2(p) + \big( a(p) + 1 \big)^2 \right) t^2 = B t^2,
  \end{multline*}
  где $B := \frac{4}{\alpha^2} \left( \sigma^2(p) + \big( a(p) + 1 \big)^2 \right)$.
\end{proof}

Заметим, что для любой невырожденной (не равной константе почти наверное) случайной величины $X \geqslant 0$ найдется такое $\alpha > 0$, что $\Prob(X > \alpha) = p \in (0, 1)$. Тогда построим процесс $Z^\star$, как в определении \ref{dfstar}, по независимым одинаково распределенным случайным величинам
\begin{equation*}
  Y_n =
  \begin{cases}
    \alpha, &\text{если }X_n > \alpha,\\
    0, &\text{если }X_n \leqslant \alpha.
  \end{cases}
\end{equation*}

По построению $Y_n \leqslant X_n$, откуда $Z(t) \leqslant Z^\star(t)$, $t \geqslant 0$.

\begin{cor}
  $\Expect Z(t) \leqslant A t$ и $\Expect Z(t)^2 \leqslant B t^2$ для любого $t \geqslant \alpha$. В частности, $Z(t) < \infty$ п.\,н. при всех $t \geqslant 0$.
\end{cor}

\begin{cor}
  $\Prob\left( \forall\, t \geqslant 0\; Z(t) < \infty \right) = 1$.
\end{cor}

\begin{proof}
  Поскольку $Z(t)$ является неубывающим процессом, т.\,е. $\forall\, s \leqslant t \; Z(s) \leqslant Z(t)$, то достаточно доказать, что $\Prob\left(\forall\,n \in \nat\; Z(n) < \infty\right) = 1$. Но
  \begin{equation*}
    \left\{ \forall\,n \in \nat\; Z(n) < \infty \right\} = \bigcap_{n \in \nat} \left\{ Z(n) < \infty \right\} \text{"---}
  \end{equation*}
  счетное пересечение событий вероятности 1 (см. предыдущее следствие). Оно тоже имеет вероятность 1.
\end{proof}


\subsection{Элементарная теория восстановления}

\begin{lem}
  Пусть $X, X_1, X_2, \ldots$  "--- н.\,о.\,р. случайные величины, $X \geqslant 0$. Тогда $\frac{S_n}{n} \as \mu \in [0, \infty]$ при $n \to \infty$, где $\mu = \Expect X$ (конечное или бесконечное).
\end{lem}

\begin{proof}
  Если $\mu < \infty$, то утверждение леммы представляет собой усиленный закон больших чисел А.\,Н.\,Колмогорова.

  Пусть $\mu = \infty$. Положим для $c > 0$
  \begin{equation*}
    V_n(c) := X_n \ind\left\{X_n \leqslant c\right\}.
  \end{equation*}
  Тогда снова по УЗБЧ А.\,Н.\,Колмогорова $\frac{1}{n} \sum\limits_{k = 1}^n V_k \as \Expect X \ind\left\{X_n \leqslant c\right\}$.

  Возьмем $c = m \in \nat$. Тогда с вероятностью 1
  \begin{equation*}
    \liminf_{n\to \infty} \frac{1}{n}\sum_{k = 1}^n X_k \geqslant \lim_{m \to \infty} \Expect X \ind\left\{X \leqslant m\right\} = \Expect X.
  \end{equation*}
  В последнем равенстве использовалась теорема о монотонной сходимости (для бесконечного предельного интеграла).
\end{proof}

Введем определение, которое понадобится нам в дальнейшем.

\begin{df}
  Семейство случайных величин $\left\{ \xi_\alpha, t \in \Lambda \right\}$ называется \emph{равномерно интегрируемым}, если
  \begin{equation*}
    \lim_{c \to \infty} \sup_{\alpha \in \Lambda} \int_{ \left\{ |\xi_\alpha| \geqslant c \right\}} |\xi_\alpha| \dif\Prob = 0.
  \end{equation*}
\end{df}

Известно, что если семейство $\left\{ \xi_n, n \geqslant 1\right\}$ равномерно интегрируемо и $\xi_n \to \xi$ почти наверное, то $\xi$ тоже интегрируема и $\Expect \xi_n \to \Expect \xi$. Для неотрицательных случайных величин $\xi_n$, $n \geqslant 1$, таких, что $\xi_n \to \xi$ п.\,н., где $\Expect \xi < \infty$, имеет место и обратная импликация
\begin{equation*}
  \Expect \xi_n \to \Expect \xi\; \Longrightarrow\; \text{семейство $\left\{ \xi_n, n \geqslant 1 \right\}$ равномерно интегрируемо.}
\end{equation*}

Следующая теорема принимается без доказательства
\begin{thm}[Де ла Валле Пуссен]\index{Теорема!Де ла Валле Пуссена}\label{pussen}
  \sloppy
  Семейство случайных величин $\left\{ \xi_\alpha, \alpha \in \Lambda\right\}$ является равномерно интегрируемым тогда и только тогда, когда найдется измеримая функция $g\colon \real_+ \to \real_+$, т.\,е. $g \in \borel(\real_+) | \borel(\real_+)$, такая, что
  \begin{equation*}
    \lim_{t \to \infty}\frac{g(t)}{t} = \infty\quad \text{и}\quad \sup \Expect g(|\xi_\alpha|) < \infty.
  \end{equation*}
\end{thm}

\begin{thm}
  Пусть $Z = \left\{ Z(t), t\geqslant 0 \right\}$ "--- процесс восстановления, построенный по последовательности н.\,о.\,р случайных величин $X, X_1, X_2, \ldots$ . Тогда
  \begin{enumerate}
    \item\label{firstel} $\displaystyle \frac{Z(t)}{t} \as \frac{1}{\mu}$ при $t \to \infty$;
    \item\label{secondel} $\displaystyle \frac{\Expect Z(t)}{t} \to \frac{1}{\mu}$ при $t \to \infty$, где $\frac{1}{0} := \infty$, $\frac{1}{\infty} := 0$.
  \end{enumerate}
\end{thm}

\begin{proof}
  Если $\mu = 0$, то $X_n = 0$ п.\,н., поэтому $\forall\, t > 0 \; Z(t) = \infty$ и утверждение теоремы очевидно.

  Далее $\mu > 0$. Заметим, что
  \begin{equation}
    S_{Z(t)} \leqslant t < S_{Z(t) + 1}
    \label{eqz}
  \end{equation}
  Для фиксированного $\omega$ рассмотрим последовательность $t_n := S_n(\omega)$. Поскольку $Z(t_n, \omega) = n$ и траектория $Z(t, \omega)$ монотонна, $Z(t, \omega) \to \infty$. Будем рассматривать те $(t, \omega)$, для которых $0 < Z(t, \omega) < \infty$ (при всех $t_n$, а значит, вообще при всех $t$ это выполнено почти наверное). Для этих $(t, \omega)$ разделим обе части \ref{eqz} на $Z(t)$. Получим
  \begin{equation*}
    \frac{S_{Z(t)}}{Z(t)} \leqslant \frac{t}{Z(t)} < \frac{S_{Z(t) + 1}}{Z(t) + 1}\frac{Z(t) + 1}{Z(t)}.
  \end{equation*}
  Но поскольку $Z(t) \to \infty$, то $\frac{S_{Z(t)}}{Z(t)} \as \mu$, $\frac{S_{Z(t) + 1}}{Z(t) + 1} \as \mu$ и $\frac{Z(t) + 1}{Z(t)} \to 1$. Следовательно, $\frac{t}{Z(t)} \as \mu$ при $t \to \infty$, т.\,е. $\frac{Z(t)}{t} \as \frac{1}{\mu}$, что завершает доказательство утверждения \ref{firstel}.

  Для доказательства утверждения \ref{secondel} используем теорему \ref{pussen}. А именно, рассмотрим семейство $\left\{\xi_t, t \geqslant \alpha\right\}$ и функцию $g(t) = t^2$, где $\xi_t = \frac{Z(t)}{t}$. По лемме \ref{est}
  \begin{equation*}
    \Expect \xi_t^2 = \frac{\Expect Z(t)^2}{t^2} \leqslant \frac{B t^2}{t^2} = B < \infty.
  \end{equation*}
  Все условия теоремы \ref{pussen} выполнены. Поэтому из нее вытекает, что семейство $\left\{\xi_t, t \geqslant \alpha\right\}$ равномерно интегрируемо. Тогда можно совершить предельный переход под знаком математического ожидания, и из утверждения \ref{firstel} получаем, что
  \begin{equation*}
    \Expect \frac{Z(t)}{t} \to \Expect \frac{1}{\mu} = \frac{1}{\mu},\quad t \to \infty.
  \end{equation*}
\end{proof}


\subsection{Пуассоновский процесс как процесс восстановления}

\begin{df}
  \sloppy
  Пусть $X$, $X_1$, $X_2$, \ldots  "--- независимые одинаково распределенные случайные величины с экспоненциальным распределением $X\sim \Exp(\lambda)$, т.\,е.
  \begin{equation*}
    p_X(x) =
    \begin{cases}
      \lambda e^{- \lambda x}, &\text{если $x \geqslant 0$,}\\
      0, &\text{если $x < 0$}.
    \end{cases}
  \end{equation*}
  \emph{Пуассоновским процессом}\index{Процесс!пуассоновский}\index{Пуассоновский процесс} $N = \left\{N(t), t\geqslant 0\right\}$ называется процесс восстановления, построенный по $X_1$, $X_2$, \ldots.
\end{df}

Для $t > 0$ введем случайные величины
\begin{align*}
  X_1^t &:= S_{N(t) + 1} - t;\\
  X_k^t &:= S_{N(t) + k},\quad k \geqslant 2.
\end{align*}

\begin{lem}
  \sloppy
  Для любого $t > 0$ случайные величины $N(t)$, $X_1^t$, $X_2^t$, \ldots являются независимыми, причем $N(t) \sim \Pois(\lambda t)$, $X_k^t \sim \Exp(\lambda)$ для $k = 1, 2, \ldots$.
\end{lem}

\begin{proof}
  Чтобы доказать независимость указанных случайных величин, достаточно проверить, что для $\forall\, n \in \nonneg\; \forall\,u_1, \ldots, u_k \geqslant 0$ выполнено
  \begin{equation*}
    \Prob(N(t) = n, X_1^t > u_1, \ldots, X_k^t > u_k) = \Prob(N(t) = n) \Prob(X_1^t > u_1) \ldots \Prob(X_k^t > u_k).
  \end{equation*}

  Доказываем это индукцией по $k$.

  База индукции: $k = 1$. Напомним (было в курсе теории вероятностей), что случайная величина $S_n$ имеет плотность
  \begin{equation*}
    p_{S_n}(x) =
    \begin{cases}
      \frac{\lambda (\lambda x)^{n - 1}}{(n - 1)!} e^{- \lambda x}, &\text{если $x \geqslant 0$};\\
      0, &\text{если $x < 0$}.
    \end{cases}
  \end{equation*}

  Итак,
  \begin{multline*}
    \Prob(N(t) = n, X_1^t > u_1) = \Prob(S_n \leqslant t, S_{n + 1} > t, S_{N(t) + 1} - t > u_1) =\\
    = \Prob(S_n \leqslant t, S_{n + 1} > t, S_{n + 1} > t + u_1) = \Prob(S_n \leqslant t, S_{n + 1} > t + u_1) =\\
    = \Prob(S_n \leqslant t, S_n + X_{n + 1} > t + u_1) =\\
    = \Prob\left((S_n, X_{n + 1}) \in \left\{(x, y)\colon x \leqslant t, x + y > t + u_1\right\} \right) =\\
    \iint\limits_{\substack{x \leqslant t\\ x + y > t + u_1}} p_{(S_n, X_{n + 1})}(x, y) \dif x \dif y = \left(\text{независимость $S_n$ и $X_{n + 1}$}\right) =\\
    = \iint\limits_{\substack{x \leqslant t\\ x + y > t + u_1}} p_{S_n}(x) p_{X_{n + 1}}(y) \dif x \dif y = \iint\limits_{\substack{0 \leqslant x \leqslant t, y \geqslant 0\\ x + y > t + u_1}} \frac{\lambda (\lambda x)^{n - 1}}{(n - 1)!} e^{- \lambda x} \lambda e^{-\lambda y} \dif x \dif y =\\
    = \left(\text{теорема Фубини}\right) = \int\limits_{0}^{t} \frac{\lambda (\lambda x)^{n - 1}}{(n - 1)!} e^{- \lambda x} \dif x \int\limits_{t + u_1 - x}^{+\infty} \lambda  e^{-\lambda y} \dif y =\\
    = \int\limits_{0}^{t} \frac{\lambda (\lambda x)^{n - 1}}{(n - 1)!} e^{- \lambda x} e^{-\lambda (t + u_1 - x)} \dif x = e^{-\lambda (t + u_1)} \int\limits_{0}^{t} \frac{\lambda (\lambda x)^{n - 1}}{(n - 1)!} \dif x =\\
    = \frac{(\lambda t)^n}{n!} e^{-\lambda t} e^{-\lambda u_1}.
  \end{multline*}

  Положим $u_1 = 0$, получим
  \begin{equation*}
    \Prob(N(t) = n, X_1^t > 0) = \Prob(N(t) = n) = \frac{(\lambda t)^n}{n!} e^{-\lambda t},\quad n\in \nonneg,
  \end{equation*}
  т.\,е. $N(t) \sim \Pois(\lambda t)$. Далее,
  \begin{equation*}
    \Prob(X_1^t > u_1) = \sum_{n = 0}^\infty \Prob(N(t) = n, X_1^t > u_1) = \sum_{n = 0}^\infty \frac{(\lambda t)^n}{n!} e^{-\lambda t} \cdot e^{-\lambda u_1} = 1\cdot e^{-\lambda u_1},
  \end{equation*}
  т.\,е. $X_1^t \sim \Exp(\lambda)$ и база установлена.

  Индукционный переход: пусть $k \geqslant 2$.
  \begin{multline*}
    \Prob(N(t) = n, X_1^t > u_1, \ldots, X_k^t > u_k) =\\ \Prob(S_n \leqslant t, S_{n + 1} > t, S_{n + 1} > t + u_1, X_{n + 2} > u_2, \ldots, X_{n + k} > u_k) =\\
    = \left(\text{см. предыдущее}\right) = \Prob(N(t) = n) \Prob(X_1^t > u_1) e^{-\lambda u_2} \ldots e^{-\lambda u_k} =\\
    = \Prob(N(t) = n)  e^{-\lambda u_1} \ldots e^{-\lambda u_k}.
  \end{multline*}

  Снова положим $u_1 = \ldots = u_{k - 1} = 0$ и просуммируем по всем $n \in \nonneg$. Получим $\Prob(X_k^t > u_k) = e^{-\lambda u_k}$, откуда $X_k^t \sim \Exp(\lambda)$, индукционный переход завершен.
\end{proof}

Пусть $X_j \sim \Exp(\lambda)$ "--- интервалы между временами прихода автобусов на данную остановку. Тогда случайная величина $X_1^t = S_{N(t) + 1} - t$ соответствует времени ожидания прибытия ближайшего автобуса. Мы только что доказали, что она распределена так же, как и интервалы: $X_1^t \sim \Exp(\lambda)$. Мы будем в среднем ждать автобуса столько же времени, сколько в среднем проходит времени между двумя автобусами. В этом состоит \textbf{парадокс времени ожидания}\index{Парадокс времени ожидания}. Никакого противоречия здесь на самом деле нет, так как сами моменты прихода автобусов также случайные.

\clearpage
\phantomsection
\addcontentsline{toc}{section}{Предметный указатель}
\printindex

\end{document}

\begin{thebibliography}{20}
%FIXME
\bibitem{B-Sh} Булинский, Ширяев,
		Т.Сл.Пр.	
\end{thebibliography}

\end{document}

 % TOC и Предметный указатель
\chapter{Случайные блуждания}

\section{Понятие случайного блуждания}

\begin{df}\index{Измеримое!пространство}
	Пусть $V$ \td множество, а $\As$ \td $\si$-алгебра его подмножеств.
	Тогда $(V, \As)$ называется \textit{измеримым пространством}.
\end{df}

\begin{df}\index{Измеримое!отображение}
 	Пусть есть $(V, \As)$ и $(S, \Bs)$ \td два измеримых пространства, 
	$f \cln V \to S$ \td отображение.
	$f$ называется \textit{$\As \divs \Bs$\hизмеримым}, если $\fA B \in \Bs \; f^{-1}(B) \in \As$.
\end{df}
\begin{denote}
	$f\in \As \divs \Bs$.
\end{denote}

\begin{df}\index{Случайный!элемент}
	Пусть есть $(\Om, \Fs, \Pf)$ \td вероятностное пространство,
	$(S, \Bs)$ \td измеримое пространство,
	$Y \cln \Om \to S$ \td отображение.
	Если $Y \in \Fs \divs \Bs$, то $Y$ называется \textit{случайным элементом}.
\end{df}

\begin{ex}
	$S = \R^m$, $\Bc = \Bc(\R)$ \td борелевские множества.
	Тогда при $m > 1$ случайный элемент $Y$ \td случайный вектор;
	если $m = 1$, то $Y$ \td случайная величина.
	$\Pf_Y(B) = \Pf[Y^{-1}(B)]$ \td мера на $\Bc$.

	Легко видеть, что
	\[
		\Pf_Y(B) = \Pf\hc{\om \in \Om \mid Y(\om) \in B}
	\]
\end{ex}

\begin{df}\index{Распределение!случайного элемента}
 	Пусть $(\Om, \Fs, \Pf)$ \td вероятностное пространство,
	$(S, \Bs)$ \td измеримое пространство,
	$Y \cln \Om \to S$ \td случайный элемент.
	\textit{Распределение вероятностей, индуцированное случайным элементом $Y$,}
	\td это функция на множествах из $\Bs$, задаваемая равенством
	\[
		\Pf_Y (B)\deq \Pf(Y^{-1}(B)), \quad B\in\mathscr{B}.
	\]
\end{df}

\begin{df}\index{Случайный!процесс}
	Пусть $(S_t, \Bs_t)_{t \in T}$ \td семейство измеримых пространств.
	\textit{Случайный процесс, ассоциированный с этим семейством,} \td это семейство случайных элементов
	$X = \hc{X(t) \mid t \in T}$, где $X(t) \cln \Om \to S_t$,
	$X(t) \in \Fs \divs \Bs_t \; \fA t \in T$.
	Здесь $T$ \td это произвольное параметрическое множество,
	$(S_t, \Bs_t)$ \td произвольные измеримые пространства.
\end{df}

\begin{note}
	Если $T \subs \R$, то $t \in T$ интерпретируется как время.
	Если $T = \R$, то время \textit{непрерывно};
	если $T = \Z$ или $T = \Z_+$, то время \textit{дискретно};
	если $T \subs \R^d$, то говорят о \textit{случайном поле}.
\end{note}

\begin{df}
	Случайные элементы $X_1 \sco X_n$ называются \textit{независимыми}, если
	\[
		\Pf \hr{\capkun \hc{X_k \in B_k}} = \prodl{k=1}{n} \Pf(X_k \in B_k),
			\quad \fA B_1 \in \Bs_1 \sco B_n \in \Bs_n.
	\]
\end{df}

\begin{theorem}[Ломницкого-Улама]\index{Теорема!Ломницкого-Улама}
 	Пусть $(S_t, \Bs_t, \Q_t)_{t \in	T}$ \td семейство вероятностных пространств.
	Тогда на некотором $(\Om, \Fs, \Pf)$ существует семейство \textit{независимых} случайных элементов
	$X_t \cln \Om \to S_t$, $X_t \in \Fs \divs \Bs_t$ таких, что $\Pf_{X_t} = \Q_{t}$, $t \in T$.
\end{theorem}

\begin{note}
	Это значит, что на некотором вероятностном пространстве можно
	задать независимое семейство случайных элементов с наперед указанными распределениеми.
	При этом $T$ по-прежнему любое, как и $(S_t, \Bs_t, \Q)_{t \in T}$ \td произвольные вероятностные пространства.
	Независимость здесь означает независимость в совокупности для любого конечного поднабора.
\end{note}

\section{Случайные блуждания}

\begin{df}\index{Случайное блуждание}
	Пусть $X, X_1, X_2\etc$ \td независимые одинаково распределенные случайные векторы	со значениями в $\R^d$.
	\textit{Случайным блужданием в $\R^d$} называется случайный процесс с дискретным временем
	$S = \hc{ S_n, n \ge 0}, n \in \Z_+$ такой, что
	\begin{align*}
			S_0 & \deq x \in \R^d \quad\text{(начальная точка)};\\
			S_n & \deq x + X_1 \spl X_n, \quad n \in \N.
	\end{align*}
\end{df}

\begin{df}\index{Случайное блуждание!простое}
	\textit{Простое случайное блуждание в $\Z^d$} \td это такое случайное блуждание, что
	\[
		\Pf(X = e_k) = \Pf(X = -e_k) = \frac1{2d},
	\]
	где $e_k = (0 \sco 0, \ub{1}_k, 0 \sco 0)$, $k = 1 \sco d$.
\end{df}

\begin{df}\index{Случайное блуждание!простое!возвратное}
	Введем $N \deq \sumnzi \Ibb \hc{S_n = 0} \le \bes$.
	Это, по сути, число попаданий нашего процесса в точку $0$.
	Простое случайное блуждание $S = \hc{S_n, n \ge 0}$ называется \textit{возвратным},
	если $\Pf(N = \bes) = 1$; \textit{невозвратным}, если $\Pf(N < \bes) = 1$.
\end{df}

\begin{note}
	Следует понимать, что хотя определение подразумевает, что $\Pf(N = \bes)$ равно либо 0, либо 1,
	пока что это является недоказанным фактом.
	Это свойство будет следовать из следующей леммы.
\end{note}
\begin{df}
	Число $\tau \deq \inf\hc{ n \in \N : S_{n} = 0}$ ($\tau \deq \bes$,
	если $S_{n} \ne 0 \; \forall\, n \in N$) называется \textit{моментом первого возвращения в 0}.
\end{df}

\begin{lemma}
	Для $ \fA n \in \N \; \Pf(N = n) = \Pf(\tau = \bes)\Pf(\tau < \bes)^{n-1}$.
		\footnote{\textit{от наборщика:} Судя по всему, в лемме подразумевается,
			что начальная точка нашего случайного блуждания \td это 0.}
\end{lemma}

\begin{proof}
	При $n = 1$ формула верна: $\hc{ N = 1} = \hc{ \tau = \bes}$.
Докажем по индукции.

	\begin{gather*}
		\Pf(N = n+1, \tau < \bes) =
		\sumkun \Pf(N = n+1, \tau = k) 
	=	\sumkun \Pf\hr{ \sum_{m=0}^{\bes} \Ibb \hc{ S_{m+k} - S_k = 0 } = n, \tau = k} = \\
	=	\sumkun \Pf \hr{ \sum_{m=0}^{\bes} \Ibb \hc{ S_m = 0} = n} \Pf(\tau = k) 
	=	\sumkun \Pf(N^{\prime} = n)\Pf(\tau = k),
	\end{gather*}
	где $N^{\prime}$ определяется по последовательности
	$X_1^{\prime} = X_{k+1}, X_{2}^{\prime} = X_{k+2}$ и так далее.
	Из того, что $X_i$ \td независиые одинаково распределенные случайные векторы,
	следует, что $N^{\prime}$ и $N$ распределены одинаково.
	Таким образом, получаем, что
	\[
		\Pf(N = n+1, \tau < \bes) = \Pf(N = n)\Pf(\tau < \bes).
	\]
	Заметим теперь, что
	\[
		\Pf(N = n+1) = \Pf(N = n+1, \tau < \bes) + \Pf(N = n+1, \tau = \bes),
	\]
	где второе слагаемое обнуляется из-за того, что $n+1 \ge 2$.
	Из этого следует, что
	\[
		\Pf(N = n+1) = \Pf(N = n)\Pf(\tau < \bes).
	\]
		Пользуемся предположением индукции и получаем, что
	\[
		\Pf(N = n+1) = \Pf(\tau = \bes)\Pf(\tau < \bes)^n,
	\]
	что и завершает доказательство леммы.
\end{proof}

\begin{imp}
	$\Pf(N = \bes)$ равно 0 или 1.
	$\Pf(N < \bes) = 1 \Lra \Pf(\tau < \bes) < 1$.
\end{imp}

\begin{proof}
	Пусть $\Pf(\tau < \bes) < 1$.
	Тогда
	\[
		\Pf(N < \bes) = \sumnui \Pf(N = n) = \sumnui \Pf(\tau = \bes) \Pf(\tau < \bes)^{n-1}
		= \frac{\Pf(\tau = \bes)}{1 - \Pf(\tau < \bes)} = \frac{\Pf(\tau = \bes)}{\Pf(\tau = \bes)} = 1.
	\]
	Это доказывает первое утверждение следствия и импликацию справа налево в формулировке следствия.
	Докажем импликацию слева направо.
	\[
		\Pf(\tau < \bes) = 1 \Ra \Pf \hr{\tau = \bes } = 0 \Ra \Pf(N = n) = 0
			\quad \fA n \in \N \Ra \Pf(N < \bes) = 0.
	\]
	Следствие доказано.
\end{proof}

\begin{theorem}
	Простое случайное блуждание в $\Z^d$ возвратно $\Lra$ $\Ef N = \bes$
	(соответственно, невозвратно $\Lra$ $\Ef N < \bes$).
\end{theorem}

\begin{proof}
	Если $\Ef N < \bes$, то $\Pf(N<\bes) = 1$.
	Пусть теперь $\Pf(N<\bes) = 1$.
	Это равносильно тому, что $\Pf(\tau < \bes) < 1$.
	\[
		\Ef N = \sumnui n\Pf(N=n)
	=	\sumnui n\Pf(\tau = \bes)\Pf(\tau < \bes)^{n-1}
	=	\Pf(\tau = \bes)\sumnui n\Pf(\tau < \bes)^{n-1}.
	\]
	Заметим, что
	\[
		\sumnui np^{n-1} = \hr{\sumnui p^n}^{\prime} = (\frac1{1-p})^{\prime} = \frac1{(1-p)^2}.
	\]
	Тогда, продолжая цепочку равенств, получаем, что
	\[
		\Pf(\tau = \bes)\sumnui n\Pf(\tau < \bes)^{n-1} =
		\frac{\Pf(\tau = \bes)}{(1 - \Pf(\tau < \bes))^2} = \frac1{1 - \Pf(\tau < \bes)},
	\]
	что завершает доказательство теоремы.
\end{proof}

\begin{note}
	Заметим, что поскольку $N = \sumnui \Ibb \hc{S_n = 0}$, то
	\[
		\Ef N = \sumnzi \Ef \Ibb \hc{S_n = 0} = \sumnzi \Pf(S_n = 0),
	\]
	где перестановка местами знаков матожидания и суммы возможна в силу неотрицательности членов ряда.
	Таким образом,
	\begin{center}
		S возвратно $\Lra$ $\sumnzi \Pf(S_n = 0) = \bes$.
	\end{center}
\end{note}

\begin{imp}
	$S$ возвратно при $d = 1$ и $d = 2$.
\end{imp}

\begin{proof}
	\[
		\Pf(S_{2n} = 0) =
		(\frac1{2d})^{2n} \sum_{\substack{n_1 \sco n_d \ge 0 \\ n_1 \spl n_d = n}} \frac{(2n)!}{(n_1!)^2 \ldots (n_d!)^2}.
	\]
	\textit{Случай d = 1}: $\Pf(S_{2n} = 0) = \frac{(2n)!}{(n!)^{2}}(\frac{1}{2})^{2n}$.
	Согласно формуле Стирлинга,
	\[
		m! \sim \hr{\frac m e}^m \sqrt{2 \pi m}, \quad m \to \bes.
	\]
	Соответственно,
	\[
		\Pf(S_{2n} = 0) \sim \frac{1}{\sqrt{\pi n}} \Ra
	\]
	ряд $\sumnzi \frac 1{\sqrt{\pi n}} = \bes \Ra$ блуждание возвратно.
	Аналогично рассматривается \textit{случай d = 2}:
	\[
		\Pf(S_{2n} = 0) = \ldots = \hc{ \frac{(2n)!}{(n!)^2}(\frac 1{2})^{2n} }^{2} \sim \frac{1}{\pi n} \Ra
	\]
	ряд тоже разойдется $\Ra$ блуждание возвратно.
	Теорема доказана.
\end{proof}

\section{Исследование случайного блуждания с помощью характеристической функции}

\begin{theorem}
	Для простого случайного блуждания в $\Z^{d}$
	\[
		\Ef N = \lim_{c \up 1} \frac 1{(2 \pi)^d} \ints{[-\pi, \pi]^d} \frac1{1 - c \phi (t)}\diff{t},
	\]
	где $\phi (t)$ \td характеристическая функция X, $t \in \mathbb{R}^{d}$.
\end{theorem}

\begin{proof}
	\[
		\ints{[-\pi, \pi]} \frac{e^{inx}}{2 \pi} \diff{x} = \bcase{
				1, && n &= 0, \\
				0, && n &\ne 0.
		}
	\]
	Следовательно,
	\[
		\Ibb \hc{S_n = 0}
	=	\prod_{k=1}^d \Ibb \hc{ S_n^{(k)} = 0}
	=	\prod_{k=1}^d \ints{[-\pi, \pi]} \frac{e^{i S_n^{(k)} t_k}}{2 \pi}\diff{ t_k}
	=	\frac1{(2 \pi)^d} \ints{[-\pi, \pi]^d} e^{i (S_n, t)}\diff{ t}.
	\]
	По теореме Фубини
	\[
		\Ef \Ibb \hc{S_n = 0}
	=	\Ef \frac1{(2 \pi)^d} \ints{[-\pi, \pi]^d} e^{i (S_n, t)}\diff{ t}
	=	\frac1{(2 \pi)^d} \ints{[-\pi, \pi]^d} \Ef e^{i (S_n, t)}\diff{ t}.
	\]
	Заметим, что
	\[
		\Ef e^{i (S_n, t)} = \prod_{k=1}^n \phi_{X_k} (t) = (\phi (t))^n.
	\]
	Тогда
	\[
		\Ef \Ibb (S_n = 0) = \Pf(S_n = 0)
	=	\frac1{(2 \pi)^d}\ints{[-\pi, \pi]^d} \hr{\phi (t)}^n\diff{ t}.
	\]
	Из этого следует, что
	\[
		\sumnzi c^n \Pf(S_n = 0)
	=	\frac1{(2 \pi)^d} \ints{[-\pi, \pi]^d} \sumnzi (c \phi(t))^n\diff{t},\quad\text{где $0 < c < 1$}.
	\]
	Поскольку $|c \phi| \le c < 1$, то
	\[
		\frac1{(2 \pi)^d} \ints{[-\pi, \pi]^d} \sumnzi (c \phi(t))^n\diff{ t}
	=	\frac1{(2 \pi)^d} \ints{[-\pi, \pi]^d} \frac1{1 - c \phi (t)}\diff{ t}
	\]
	по формуле для суммы бесконечно убывающей геометрической прогрессии.
	Осталось только заметить, что
	\[
		\sumnzi c^n \Pf(S_n = 0) \to \sumnzi \Pf(S_n = 0) = \Ef N, \quad c \uparrow 1,
	\]
	что и завершает доказательство теоремы.
\end{proof}

\begin{imp}
	При $d \ge 3$ простое случайное блуждание невозвратно.
\end{imp}

\begin{note}
	Можно говорить и о случайных блужданиях в $\R^d$, если $X_i: \Om \to \R^d$.
	Но тогда о возвратности приходится говорить в терминах
	бесконечно частого попадания в $\epsilon$\hокрестность точки $x$.
\end{note}

\begin{df}\index{Множество!возвратности}
	Пусть есть случайное блуждание $S$ на $\R^d$.
	Тогда \textit{множество возвратности} случайного блуждания $S$ \td это множество
	\[
		R(S) = \hc{ x \in \R^d : \text{блуждание возвратно в окрестности точки } x}.
	\]
\end{df}

\begin{df}\index{Множество!достижимости}
	Пусть есть случайное блуждание $S$ на $\R^d$.
	Тогда \textit{точки, достижимые случайным блужданием $S$,} \td это множество $P(S)$ такое, что
	\[
		\fA z \in P(S) \; \fA \epsilon > 0 \; \Ex n : \; \Pf( | S_{n} - z | < \epsilon) > 0.
	\]
\end{df}

\begin{theorem}[Чжуна-Фукса]\index{Теорема!Чжуна-Фукса}
	Если $R(S) \neq \emptyset$, то $R(S) = P(S)$.
\end{theorem}

\begin{imp}
	Если $0 \in R(S)$, то $R(S) = P(S)$;
	если $0 \notin R(S)$, то	$R(S) = \emptyset$.
\end{imp}

\chapter{Ветвящиеся процессы и процессы восстановления}

\section{Модель Гальтона\DВатсона}\index{Модель Гальтона-Ватсона}

\paragraph{Описание модели.}

Пусть $\hc{ \xi, \xi_{n, k} \mid n, k \in \N}$ \td массив независимых одинаково распределенных случайных величин,
\[
	\Pf (\xi = m) = p_m \ge 0, \; m \in \Z_{+} = \hc{0, 1, 2\etc}.
\]
Такие существуют в силу теоремы Ломницкого--Улама.
Положим
\begin{align*}
	Z_0(\omega) &\deq 1,\\
	Z_n(\omega) &\deq \sum_{k = 1}^{Z_{n - 1}(\omega)} \xi_{n, k}(\omega) \quad \text{для $n \in \N$}.
\end{align*}
Здесь подразумевается, что если $Z_{n-1}(\omega) = 0$, то и вся сумма равна нулю.
Таким образом, рассматривается сумма случайного числа случайных величин.
Определим
$A = \hc{\omega \mid \Ex n = n(\omega): Z_{n}(\omega) = 0}$ \td \index{Вырождение}\textit{событие вырождения популяции}.
Заметим, что если $Z_n(\omega) = 0$, то $Z_{n+1}(\omega) = 0$.
Таким образом,
$\hc{ Z_n = 0} \subs \hc{ Z_{n+1} = 0}$ и $A = \cupnui \hc{ Z_{n} = 0}$.

По свойству непрерывности вероятностной меры,
\[
	\Pf(A) = \lim_{n \to \bes} \Pf(Z_n = 0).
\]

\begin{df}\index{Производящая функция}
	Пусть дана последовательность $(a_n)_{n=0}^{\bes}$ неотрицательных чисел такая,
	что $\sumnzi a_n = 1$.
	\textit{Производящая функция} для этой последовательности \td это
	\[
		f(s) \deq \sumkzi s^k a_k, \quad |s| \le 1
	\]
	(нас в основном будут интересовать $s \in [0, 1]$).
\end{df}

Заметим, что если $a_k = \Pf(Y = k), k = 0, 1,\; \ldots$ , то
\[
	f_Y(s) = \sumkzi s^k \Pf(Y = k) = \Ef s^Y, \quad s \in [0, 1].
\]

\begin{lemma}
	\label{lem1}
	Вероятность $\Pf(A)$ является корнем уравнения $\psi(p) = p$, где $\psi = f_{\xi}$ и $p \in [0, 1]$.
\end{lemma}

\begin{proof}
	\begin{multline*}
		f_{Z_n}(s)
	=	\Ef s^{Z_n}
	=	\Ef \hr{s^{\sum_{k = 1}^{Z_{n - 1}} \xi_{n, k}}} = \\
	=	\sumjzi \Ef \hs{ \hr{ s^{\sum_{k = 1}^{Z_{n - 1}} \xi_{n, k}} } \Ibb \hc{ Z_{n - 1} = j}} = \\
	=	\sumjzi \Ef \hs{ \hr{ s^{\sum_{k = 1}^j \xi_{n, k}} } \Ibb \hc{ Z_{n - 1} = j} }.
	\end{multline*}
	Поскольку $\sigma \hc{ Z_r} \subs \sigma \hc{ \xi_{m, k}, \; m = 1 \sco r,\; k \in \N}$,
	которая независима с $\sigma \hc{\xi_{n, k}, \; k \in \N}$
	(строгое и полное обоснование остается в качестве упражнения), то
	\begin{multline*}
		\sumjzi \Ef \hs{ \hc{ s^{\sum_{k = 1}^j \xi_{n, k}} } \Ibb \hc{ Z_{n-1} = j } }
	=	\sumjzi \Ef \hr{s^{\sum_{k = 1}^j \xi_{n, k}}} \Ef \Ibb \hc{ Z_{n - 1} = j} = \\
	=	\sumjzi \Ef \hr{s^{\sum_{k=1}^j \xi_{n, k}}} \Pf ( Z_{n - 1} = j )
	=	\sumjzi \prod_{k = 1}^j \Ef s^{\xi_{n, k}} \Pf (Z_{n - 1} = j) = \\
	=	\sumjzi \psi_{\xi}^j (s) \Pf (Z_{n - 1} = j)
	=	f_{Z_{n - 1}} \hr{\psi_{\xi} (s)}
	\end{multline*}
	в силу независимости и одинаковой распределенности $\xi_{n, k}$ и определения производящей функции.
	Таким образом,
	\[
		f_{Z_n} (s) = f_{Z_{n - 1}} \hr{ \psi_{\xi} (s)}, \quad s \in [0, 1]{.}
	\]
	Подставим $s = 0$ и получим, что
	\[
		f_{Z_n} (0) = f_{Z_{n - 1}} \hr{ \psi_{\xi} (0)}
	\]
	Заметим, что
	\[
		f_{Z_n}(s)
	=	f_{Z_{n - 1}}(\psi_{\xi}(s))
	=	f_{Z_{n - 2}} \hr{\psi_{\xi} \hr{ \psi_{\xi} (s) } }
	=	\ldots
	=	\ub{\psi_{\xi} (\psi_{\xi} \ldots (\psi_{\xi}(s)) \ldots ) }_{\text{$n$ итераций}}
	=	\psi_{\xi} (f_{Z_{n - 1}} (s)).
	\]
	Тогда при $s = 0$ имеем, что
	\[
		\Pf (Z_n = 0) = \psi_{\xi} \hr{ \Pf\hr{Z_{n - 1} = 0}}.
	\]
	Но $\Pf(Z_n = 0) \upto \Pf(A)$ при $n \to \bes$ и $\psi_{\xi}$ непрерывна на $[0, 1]$.
	Переходим к пределу при $n \to \bes$.
	Тогда
	\[
		\Pf(A) = \psi_{\xi} (\Pf(A)),
	\]
	то есть $\Pf(A)$ \td корень уравнения $p = \psi_{\xi}(p)$, $p \in [0, 1]$.
\end{proof}

\begin{theorem}
	Вероятность $p$ вырождения процесса Гальтона\DВатсона есть \textbf{наименьший} корень уравнения
	\begin{equation}
		\label{eq1}
		\psi(p) = p, \quad p \in [0, 1],
	\end{equation}
	где $\psi = \psi_{\xi}$.
\end{theorem}

\begin{proof}
	Пусть $p_0 \deq \Pf(\xi = 0) = 0$.
	Тогда
	\[
		\Pf(\xi \ge 1) = 1, \quad \Pf\hr{\caps{n,k} \hc{ \xi_{n,k} \ge 1} } = 1.
	\]
	Поэтому $Z_n \ge 1$ при $\fA n$, то есть $\Pf(A)$ \td наименьший корень уравнения~\eqref{eq1}.

	Пусть теперь $p_0 = 1$.
	Тогда $\Pf(\xi = 0)=1 \Ra \Pf(A)$ \td наименьший корень уравнения~\eqref{eq1}.

	Пусть, наконец, $0 < p_{0} < 1$.
	Из этого следует, что $\Ex m\in\N:\; p_m > 0$, а значит, $\psi$ строго возрастает на $[0, 1]$.
	Рассмотрим
	\[
		\Delta_n = \hsr{\psi_{n}(0), \psi_{n + 1}(0)},\quad n = 0, 1, 2 \etc ,
	\]
	где $\psi_n(s)$ \td это производящая функция $Z_n$.

	Пусть $s \in \Delta_n$.
	Тогда из монотонности $\psi$ на $[0, 1]$ получаем, что
	\[
		\psi(s) - s > \psi(\psi_n(0)) - \psi_{n + 1}(0)  =  \psi_{n + 1}(0) - \psi_{n + 1}(0) = 0,
	\]
	что означает, что у уравнения~\eqref{eq1} нет корней на $\Delta_n \; \fA n \in \Z_{+}$.
	Заметим, что
	\[
		\cupnzi \Delta_n = \hsr{0, \Pf(A)}, \quad \psi_n(0) \upto \Pf(A).
	\]
	По лемме \ref{lem1} $\Pf(A)$ является корнем уравнения \eqref{eq1}.
	Следовательно, показано, что $\Pf(A)$ \td наименьший корень, что и требовалось доказать.
\end{proof}

\begin{theorem}
	\begin{points}{0}
		\item\label{firth} Вероятность вырождения $\Pf(A)$ есть нуль $\Lra$ $p_{0} = 0$.
		\item\label{secth} Пусть $p_{0} > 0$.
			Тогда при $\Ef \xi \le 1$ имеем $\Pf (A) = 1$,
			при $\Ef \xi > 1$ имеем $\Pf(A) < 1$.
	\end{points}
\end{theorem}

\begin{proof}
	Докажем \autoref{firth}.
	Пусть $\Pf(A) = 0$.
	Тогда $p_{0} = 0$, потому что иначе была бы ненулевая вероятность вымирания $\Pf(A) > \Pf(Z_1 = 0) = p_0$.
	В другую сторону, если $p_0 = 0$, то вымирания не происходит (почти наверное) из-за того,
	что у каждой частицы есть как минимум один потомок (почти наверное).

	Докажем \autoref{secth}.
	Пусть $\mu = \Ef\xi \le 1$.
	Покажем, что в таком случае у уравнения \eqref{eq1} будет единственный корень, равный $1$.
	\[
		\psi_{\xi}^{\prime} (z)
	=	\sumkui kz^{k - 1}\Pf(\xi = k) \Ra \psi_{\xi}^{\prime} (z) > 0, \quad \text{ при $z > 0$,}
	\]
	если только $\xi$ не тождественно равна нулю (в противном случае утверждение теоремы выполнено).
	Заметим также, что $\psi_{\xi}^{\prime} (z)$ возрастает на $z > 0$.
	Воспользуемся формулой Лагранжа:
	\[
		1 - \psi_{\xi} (z)
	=	\psi_{\xi} (1) - \psi_{\xi} (z)
	=	\psi_{\xi}^{\prime} (\theta) (1 - z)
	<	\psi_{\xi}^{\prime} (1) (1-z) \le 1 - z,
	\]
	где $z \in (0, 1)$, в силу монотонности $\psi_{\xi}^{\prime} (z)$.
	Следовательно, если $z < 1$, то
	\[
		1 - \psi_{\xi}(z) < 1 - z,
	\]
	то есть $z = 1$ \td это единственный корень уравнения \eqref{eq1}.
	Значит, $P(A) = 1$.

	Пусть $\mu = \Ef\xi > 1$.
	Покажем, что в таком случае у уравнения \eqref{eq1} есть два корня, один из которых строго меньше единицы.
	\[
		\psi_{\xi}^{\prime\prime}(z)
	=	\sum\limits_{k=2}^{\bes} k (k - 1) z^{k - 2} \Pf(\xi = k),
	\]
	следовательно, $\psi_{\xi}^{\prime\prime}(z)$ монотонно возрастает и больше нуля при $z > 0$.
	Из этого следует, что $1 - \psi_{\xi}^{\prime} (z)$ строго убывает, причем
	\begin{align*}
		& 1 - \psi_{\xi}^{\prime} (0) = 1 - \Pf(\xi = 1) > 0 , \\
		& 1 - \psi_{\xi}^{\prime} (1) = 1 - \mu < 0 .
	\end{align*}
	Рассмотрим теперь $z - \psi_{\xi} (z)$ при $z = 0$.
	Поскольку $1 - \psi_{\xi} (1) = 0$, производная этой функции монотонно убывает,
	а $0 - \psi_{\xi} (0) = -\Pf(\xi = 0) < 0$,
	то график функции $z - \psi_{\xi} (z)$ пересечет ось абсцисс в двух точках,
	одна из которых будет лежать в интервале $(0, 1)$.
	Так как вероятность вырождения $\Pf(A)$ равна наименьшему корню уравнения \eqref{eq1},
	то $\Pf(A) < 1$, что и требовалось доказать.
\end{proof}

\begin{imp}
	Пусть $\Ef \xi < \bes$.
Тогда $\Ef Z_{n} = (\Ef \xi)^{n},\; n \in \N{.}$
\end{imp}

\begin{proof}
	Доказательство проводится по индукции.

	База индукции: $n=1 \Ra \Ef Z_{1} = \Ef \xi$.

	Индуктивный переход:
	\[
		\Ef Z_{n}
	=	\Ef \hr{ \sum_{k = 1}^{Z_{n - 1}} \xi_{n,k} }
	=	\sumjzi j \Ef \xi \Pf(Z_{n - 1} = j)
	=	\Ef \xi \Ef Z_{n - 1}
	=	\hr{\Ef \xi}^n.
	\]
\end{proof}

\begin{df}

	При $\Ef \xi < 1$ процесс называется \textit{докритическим.}

	При $\Ef \xi = 1$ процесс называется \textit{критическим.}

	При $\Ef \xi > 1$ процесс называется \textit{надкритическим.}
\end{df}

\section{Процессы восстановления}

\begin{df}\index{Процесс!восстановления}
	Пусть $S_n = X_1 \spl X_n$, где $n \in \N$, 
	$X, X_1, X_{2} \etc$ \td независимые одинаково распределенные случайные величины, $X \ge 0$.
	Положим
	\begin{align*}
		Z(0) &\deq 0;\\
		Z(t) &\deq \sup \hc{ n \in \N \mid S_n \le t }, \quad t > 0.
	\end{align*}
	(здесь считаем, что $\sup \varnothing \deq \bes$).
	Таким образом,
	\[
		Z(t, \omega) = \sup \hc{ n \in \N \mid S_n(\omega) \le t }.
	\]
	Иными словами,
	\[
		\hc{ Z(t) \ge n} = \hc{ S_n \le t}.
	\]
	Так определенный процесс $Z(t)$ называется \textit{процессом восстановления}.
\end{df}

\begin{note}
	Полезно заметить, что
	\[
		Z(t) = \sumnui \Ibb \hc{ S_n \le t},\quad t > 0.
	\]
\end{note}
\begin{df}
%FIXME
\phantomsection
\label{dfstar}
	Рассмотрим \textit{процесс восстановления} $\hc{ Z^\star(t),\; t \ge 0 }$,
	который строится по $Y, Y_{1}, Y_{2}\etc$ \td независимым одинаково распределенным случайным величинам,
	где $\Pf(Y = \alpha) = p \in (0, 1)$, $\Pf(Y = 0) = 1 - p$.
	Исключаем из рассмотрения случай, когда $Y = C = \const$:
		если $C = 0$, то $Z(t) = \bes \; \fA t > 0$;
		если же $C > 0$, то $Z(t) = \hs{\frac t c}$.
\end{df}

\begin{lemma}
	Для $l = 0, 1, 2\etc$
	\[
		\Pf(Z^{\star}(t) = m) =
		%FIXME?
		\bcase{
			C_m^j p^{\hs{\frac t \alpha} + 1} q^{m - \hs{\frac t \alpha}}, &\quad\text{если  $m \ge j$;} \\
			0,&\quad\text{если $m < j$.}
		}
	\]
\end{lemma}

\chapter{Пуассоновские процессы}

\section{Процессы восстановления (продолжение)}

\begin{df}
	Будем говорить, что дискретная случайная величина $U$ имеет
	\index{Распределение!геометрическое}\textit{геометрическое распределение} с параметром $p \in (0, 1)$,
	если для $k = 0, 1, 2\etc$ $\Pf(U = k) = (1 - p)^k p$.
\end{df}

\begin{lemma}\label{lemsum}
	Рассмотрим независимые геометрические величины $U_0 \sco U_j$ с параметром $p \in (0, 1)$,
	где $j = \hs{\frac t \alpha}$.
Тогда
	\[
		\Pf (j + U_0 \spl U_j = m)
	=	\Pf (Z^{\star}(t) = m ).
	\]
\end{lemma}

\begin{proof}
	 Обозначим $M = \hc{ (k_0\sco k_j)\mid k_j \in \Z_+, \sum\limits_{i = 0}^j k_j = m - j}$.
	\begin{multline*}
		\Pf\hr{ U_0 \spl U_j = m - j}
	=	\sum_{(k_0 \sco k_j) \in M} \Pf(U_0 = k_0 \sco U_j = k_j) = \\
	=	\sum_{(k_0 \sco k_j) \in M} \Pf(U_0 = k_0) \sd \Pf(U_j = k_j)
	=	\sum_{(k_0 \sco k_j) \in M} p (1 - p)^{k_0} \sd p (1 - p)^{k_j} = \\
	=	\sum_{(k_0 \sco k_j) \in M} p^{j + 1} (1 - p)^{k_0 \spl k_j}
	=	p^{j + 1} (1 - p)^{m - j} \#M = C_m^j p^{j + 1} (1 - p)^{m - j}.
	\end{multline*}
\end{proof}

\section{Сопоставление исходного процесса восстановления со вспомогательным}

\begin{lemma}\label{est}
	Пусть $t \ge \alpha$.
	Тогда $\Ef Z^\star(t) \le At$ и $\Ef Z^\star(t)^2 \le B t^2$,
	где $A \deq A(p, \alpha) > 0$, $B \deq B(p, \alpha) > 0$.
\end{lemma}

\begin{proof}
	По лемме \ref{lemsum} $\Ef Z^\star(t) = \Ef(j + U_0 \spl U_j) = j + (j + 1) \Ef U$,
	где $\Ef U \eqd a(p) < \bes$ \td математическое ожидание геометрического распределения.

	Тогда
	\[
		\Ef Z^\star(t) = j + (j + 1) a(p)
	\le	(j + 1) \hr{a(p) + 1}
	\le	\frac{t + \alpha}{\alpha} \hr{a(p) + 1}
	\le	\frac{2 t}{\alpha} \hr{a(p) + 1}
	=	A t,
	\]
	%FIXME?
	где $A \deq \cfrac{2 (a(p) + 1)}{\alpha}$.

	Далее,
	\begin{multline*}
		\Ef Z^\star(t)^2
	=	\Df Z^\star(t) + \hr{\Ef Z^\star(t)}^2
	\le	(j + 1) \ub{\Df U}_{\sigma^2(p)} + (j + 1)^2 \hr{ a(p) + 1 }^2 \le\\
	\le	(j + 1)^2 \hr{ \sigma^2(p) + \hr{ a(p) + 1 }^2}
	\le	\frac 4 {\alpha^2} \hr{ \sigma^2(p) + \hr{ a(p) + 1 }^2} t^2
	=	B t^2,
	\end{multline*}
	где $B \deq \frac4{\alpha^2} \hr{ \sigma^2(p) + \hr{ a(p) + 1}^2 }$.
\end{proof}

Заметим, что для любой невырожденной (не равной константе почти наверное) случайной величины $X \ge 0$
найдется такое $\alpha > 0$, что $\Pf(X > \alpha) = p \in (0, 1)$.
%FIXME
Тогда построим процесс $Z^\star$, как в определении \hyperref[dfstar]{определении} из прошлой лекции,
по независимым одинаково распределенным случайным величинам
\[
	Y_n =
	\bcase{
		\alpha, &\text{если }X_n > \alpha,\\
		0, &\text{если }X_n \le \alpha.
	}
\]

По построению $Y_n \le X_n$, откуда $Z(t) \le Z^\star(t)$, $t \ge 0$.

\begin{imp}
	\phantomsection
	\label{prev.imp}
	$\Ef Z(t) \le A t$ и $\Ef Z(t)^2 \le B t^2$ для любого $t \ge \alpha$.
	В частности, $Z(t) < \bes$ п.\,н.
	при всех $t \ge 0$.
\end{imp}

\begin{imp}
	$\Pf\hr{ \fA t \ge 0\; Z(t) < \bes } = 1$.
\end{imp}

\begin{proof}
	Поскольку $Z(t)$ является неубывающим процессом, \ie
	$\fA s \le t \; Z(s) \le Z(t)$, то достаточно доказать,
	что $\Pf\hr{\fA n \in \N\; Z(n) < \bes} = 1$.
	Но
	\[
		\hc{ \fA n \in \N\; Z(n) < \bes}
	=	\caps{n \in \N} \hc{ Z(n) < \bes } \text{\td}
	\]
	счетное пересечение событий вероятности 1 (см. \hyperref[prev.imp]{предыдущее следствие}).
	Оно тоже имеет вероятность 1.
\end{proof}


\section{Элементарная теория восстановления}

\begin{lemma}
	Пусть $X, X_1, X_2\etc$	\td н.\,о.\,р. случайные величины, $X \ge 0$.
	Тогда $\frac{S_n}{n} \convas \mu \in [0, \bes]$ при $n \to \bes$, где $\mu = \Ef X$ (конечное или бесконечное).
\end{lemma}

\begin{proof}
	Если $\mu < \bes$, то утверждение леммы представляет собой усиленный закон больших чисел А.\,Н.\,Колмогорова.

	Пусть $\mu = \bes$.
	Положим для $c > 0$
	\[
		V_n(c) \deq X_n \Ibb\hr{X_n \le c}.
	\]
	Тогда снова по УЗБЧ А.\,Н.\,Колмогорова $\frac 1 n \sumkun V_k \convas \Ef X \Ibb \h{X_n \le c}$.

	Возьмем $c = m \in \N$.
	Тогда с вероятностью 1
	\[
		\liminf_{n \to \bes} \frac 1 n \sumkun X_k
	\ge	\lim_{m \to \bes} \Ef X \Ibb\hc{X \le m} = \Ef X.
	\]
	В последнем равенстве использовалась теорема о монотонной сходимости (для бесконечного предельного интеграла).
\end{proof}

Введем определение, которое понадобится нам в дальнейшем.

\begin{df}
	Семейство случайных величин $\hc{ \xi_\alpha, t \in \Lambda }$ называется \textit{равномерно интегрируемым}, если
	\[
		\lim_{c \to \bes} \sup_{\alpha \in \Lambda} \ints{ \hc{ |\xi_\alpha| \ge c}} |\xi_\alpha| \diff\Pf = 0.
	\]
\end{df}

Известно, что если семейство $\left\{ \xi_n, n \ge 1\right\}$ равномерно интегрируемо и $\xi_n \to \xi$ почти наверное,
то $\xi$ тоже интегрируема и $\Ef \xi_n \to \Ef \xi$.
Для неотрицательных случайных величин $\xi_n$, $n \ge 1$, таких, что $\xi_n \to \xi$ п.\,н.,
где $\Ef \xi < \bes$, имеет место и обратная импликация
\[
	\Ef \xi_n \to \Ef \xi \Ra \text{семейство $\hc{ \xi_n, n \ge 1 }$ равномерно интегрируемо.}
\]

Следующая теорема принимается без доказательства
\begin{theorem}[Де ла Валле Пуссен]\index{Теорема!Де ла Валле Пуссена}\label{pussen}
	Семейство случайных величин $\hc{ \xi_\alpha, \alpha \in \Lambda}$ является равномерно интегрируемым тогда и только тогда,
	когда найдется измеримая функция $g \cln \R_+ \to \R_+$, \ie
	$g \in \Bs(\R_+) \divs \Bs(\R_+)$, \sth
	$\lim\limits_{t \to \bes}\frac{g(t)}{t}	= \bes$ и $\sup \Ef g(|\xi_\alpha|) < \bes$.
\end{theorem}

\begin{theorem}
	Пусть $Z = \hc{ Z(t), t\ge 0 }$ \td процесс восстановления,
	построенный по последовательности н.\,о.\,р случайных величин $X, X_1, X_2\etc$ .
	Тогда
	\begin{points}{0}
		%FIXME: маленькое пн
		\item\label{firstel} $\cfrac{Z(t)}{t} \convas \cfrac{1}{\mu}$ при $t \to \bes$;
		\item\label{secondel} $\cfrac{\Ef Z(t)}{t} \to \cfrac{1}{\mu}$ при $t \to \bes$,
	\end{points}
	где $\frac{1}{0} \deq \bes$, $\frac{1}{\bes} \deq 0$.
\end{theorem}

\begin{proof}
	Если $\mu = 0$, то $X_n = 0$ п.\,н., поэтому $\fA t > 0 \; Z(t) = \bes$ и утверждение теоремы очевидно.

	Далее $\mu > 0$.
	Заметим, что
	\begin{equation}
		S_{Z(t)} \le t < S_{Z(t) + 1}
		\label{eqz}
	\end{equation}
	Для фиксированного $\omega$ рассмотрим последовательность $t_n \deq S_n(\omega)$.
	Поскольку $Z(t_n, \omega) = n$ и траектория $Z(t, \omega)$ монотонна, $Z(t, \omega) \to \bes$.
	Будем рассматривать те $(t, \omega)$, для которых $0 < Z(t, \omega) < \bes$
	(при всех $t_n$, а значит, вообще при всех $t$ это выполнено почти наверное).
	Для этих $(t, \omega)$ разделим обе части \ref{eqz} на $Z(t)$.
	Получим
	\[
		\frac{S_{Z(t)}}{Z(t)} \le \frac{t}{Z(t)} < \frac{S_{Z(t) + 1}}{Z(t) + 1}\frac{Z(t) + 1}{Z(t)}.
	\]
	Но поскольку $Z(t) \to \bes$, то $\frac{S_{Z(t)}}{Z(t)} \convas \mu$,
	$\frac{S_{Z(t) + 1}}{Z(t) + 1} \convas \mu$ и $\frac{Z(t) + 1}{Z(t)} \to 1$.
	Следовательно, $\frac{t}{Z(t)} \convas \mu$ при $t \to \bes$, \ie
	$\frac{Z(t)}{t} \convas \frac{1}{\mu}$, что завершает доказательство утверждения \autoref{firstel}.

	Для доказательства утверждения \autoref{secondel} используем \hyperref[pussen]{{теорему Валле--Пуссена}}.
	А именно, рассмотрим семейство $\left\{\xi_t, t \ge \alpha\right\}$ и функцию $g(t) = t^2$, где $\xi_t = \frac{Z(t)}{t}$.
	По лемме \ref{est}
	\[
		\Ef \xi_t^2 = \frac{\Ef Z(t)^2}{t^2} \le \frac{B t^2}{t^2} = B < \bes.
	\]
	Все условия \hyperref[pussen]{теоремы Валле--Пуссена} выполнены.

	Поэтому из нее вытекает, что семейство $\hc{\xi_t, t \ge \alpha}$ равномерно интегрируемо.
	Тогда можно совершить предельный переход под знаком математического ожидания,
	и из утверждения \autoref{firstel} получаем, что
	\[
		\Ef \frac{Z(t)}{t} \to \Ef \frac{1}{\mu} = \frac{1}{\mu},\quad t \to \bes.
	\]
\end{proof}


\section{Пуассоновский процесс как процесс восстановления}

\begin{df}
	Пусть $X$, $X_1$, $X_2$\etc \td н.\,о.\,р. случайные величины с экспоненциальным распределением $X\sim \Exp(\lambda)$,\ie
	\[
		p_X(x) =
		\bcase{
			\lambda e^{- \lambda x}, &\quad\text{если $x \ge 0$,}\\
			0, &\quad\text{если $x < 0$}.
		}
	\]
	\textit{Пуассоновским процессом}\index{Процесс!пуассоновский}\index{Пуассоновский процесс}
	$N = \hc{N(t), t\ge 0}$ называется процесс восстановления, построенный по $X_1, X_2\etc$.
\end{df}

Для $t > 0$ введем случайные величины
\begin{align*}
	X_1^t &\deq S_{N(t) + 1} - t;\\
	X_k^t &\deq S_{N(t) + k},\quad k \ge 2.
\end{align*}

\begin{lemma}
	Для любого $t > 0$ случайные величины $N(t), X_1^t, X_2^t\etc$ являются независимыми,
	причем $N(t) \sim \Pois(\lambda t)$, $X_k^t \sim \Exp(\lambda)$ для $k = 1, 2\etc$.
\end{lemma}

\begin{proof}
	Чтобы доказать независимость указанных случайных величин,
	достаточно проверить, что для $\fA n \in \Z_+\; \fA u_1 \sco u_k \ge 0$ выполнено
	\[
		\Pf\hc{N(t) = n, X_1^t > u_1 \sco X_k^t > u_k} = \Pf(N(t) = n) \cdot \Pf(X_1^t > u_1) \sd \Pf(X_k^t > u_k).
	\]

	Доказываем это индукцией по $k$.

	База индукции: $k = 1$.
	Напомним (было в курсе теории вероятностей), что случайная величина $S_n$ имеет плотность
	\[
		p_{S_n}(x) =
		\bcase{
			\frac{\lambda (\lambda x)^{n - 1}}{(n - 1)!} e^{- \lambda x}, &\quad \text{если $x \ge 0$};\\
			0, &\quad\text{если $x < 0$}.
		}
	\]

	Итак,
	\begin{gather*}
		\Pf(N(t) = n, X_1^t > u_1)
	=	\Pf(S_n \le t, S_{n + 1} > t, S_{N(t) + 1} - t > u_1)
%XXX:	=	\Pf(S_n \le t, S_{n + 1} > t, S_{n + 1} > t + u_1) =\\
	=	\Pf(S_n \le t, S_{n + 1} > t + u_1) = \\
	=	\Pf(S_n \le t, S_n + X_{n + 1} > t + u_1)
	=	\Pf\hr{ (S_n, X_{n + 1}) \in \hc{(x, y)\mid x \le t, x + y > t + u_1 } } =\\
	=	\iints{\substack{x \le t\\ x + y > t + u_1}} p_{(S_n, X_{n + 1})}(x, y) \diff x \diff y
%	\eqvl{$S_n \perp X_{n + 1}$}{40} 
	= \hr{S_n \perp X_{n + 1}} = 
		\iints{\substack{x \le t\\ x + y > t + u_1}} p_{S_n}(x) p_{X_{n + 1}}(y) \diff x \diff y = \\
	= 	\iints{\substack{0 \le x \le t, y \ge 0\\ x + y > t + u_1}}
			\frac{\la (\la x)^{n - 1}}{(n - 1)!} e^{-\la x} \la e^{-\la y} \diff x \diff y
%	\eqvl{\text{т. Фубини}}{30}
	= \text{(т. Фубини)}
	= 
		\intl{0}{t} \frac{\la (\la x)^{n - 1}}{(n - 1)!} e^{- \la x} \diff x
		\intl{t + u_1 - x}{+\bes} \la e^{-\la y} \diff y =\\
	=	\intl{0}{t} \frac{\la (\la x)^{n - 1}}{(n - 1)!} e^{-\la x} e^{-\la (t + u_1 - x)} \diff x
	=	e^{-\la (t + u_1)} \intl{0}{t} \frac{\la (\la x)^{n - 1}}{(n - 1)!} \diff x
	=	\frac{(\la t)^n}{n!} e^{-\la t} e^{-\la u_1}.
	\end{gather*}

	Положим $u_1 = 0$, получим
	\[
		\Pf(N(t) = n, X_1^t > 0) = \Pf(N(t) = n) = \frac{(\la t)^n}{n!} e^{-\la t},\quad n\in \Z_+,
	\]
	\ie $N(t) \sim \Pois(\lambda t)$.
	Далее,
	\[
		\Pf(X_1^t > u_1)
	=	\sumnzi \Pf(N(t) = n, X_1^t > u_1)
	=	\sumnzi \frac{(\la t)^n}{n!} e^{-\la t} \cdot e^{-\la u_1} = 1\cdot e^{-\la u_1},
	\]
	\ie $X_1^t \sim \Exp(\la)$ и база установлена.

	Индукционный переход: пусть $k \ge 2$.
	\begin{multline*}
		\Pf\hc{N(t) = n, X_1^t > u_1\etc, X_k^t > u_k} =\\
	=	\Pf\hc{S_n \le t, S_{n + 1} > t, S_{n + 1} > t + u_1, X_{n + 2} > u_2\etc, X_{n + k} > u_k} 
%	\eqvl{см. выше}{30} \\
	= \text{(см. выше)} = \\
	=	\Pf\hc{N(t) = n} \Pf\hc{X_1^t > u_1}\cdot e^{-\la u_2} \sd e^{-\la u_k}
	=	\Pf\hc{N(t) = n} \cdot e^{-\la u_1} \sd e^{-\la u_k}.
	\end{multline*}

	Снова положим $u_1 = \ldots = u_{k - 1} = 0$ и просуммируем по всем $n \in \Z_+$.
	Получим $\Pf(X_k^t > u_k) = e^{-\la u_k}$, откуда $X_k^t \sim \Exp(\la)$, индукционный переход завершен.
\end{proof}

Пусть $X_j \sim \Exp(\la)$ \td интервалы между временами прихода автобусов на данную остановку.
Тогда случайная величина $X_1^t = S_{N(t) + 1} - t$ соответствует времени ожидания прибытия ближайшего автобуса.
Мы только что доказали, что она распределена так же, как и интервалы: $X_1^t \sim \Exp(\la)$.
Мы будем в среднем ждать автобуса столько же времени, сколько в среднем проходит времени между двумя автобусами.
В этом состоит \textbf{парадокс времени ожидания}\index{Парадокс времени ожидания}.
Никакого противоречия здесь на самом деле нет, так как сами моменты прихода автобусов также случайные.

\begin{thebibliography}{20}
\bibitem{B-Sh} Булинский, Ширяев,
		Теория случайных процессов
\end{thebibliography}

\end{document}
