\documentclass[a4paper, 12pt]{report}
\makeindex

\usepackage[utf8]{inputenc}
\usepackage[russian]{babel}
%\usepackage{pscyr}
\usepackage{mathtools}
\usepackage{amsmath, amssymb, amsfonts}
\usepackage{graphicx}
\usepackage[update]{epstopdf}
\usepackage[nosimple,dots,thmitshape,nodiagram]{dmvn}
\usepackage[all]{xy}
\CompileMatrices

%	XXX:К сожалению, я знаю, что это делает, пацаны говорят и это правда,
%	что можно будет кликать по ссылкам, и на практике оказалось, что да.
\usepackage[unicode]{hyperref}
\hypersetup{
	debug = true,
}

\textwidth=500pt
\textheight=750pt
\oddsidemargin=20pt
\hoffset=-1.5cm
\topmargin=-25mm
\tolerance=4000

% from Boris with love
\DeclareMathOperator{\var}{\mathrm var}
\DeclareMathOperator{\Exp}{\mathrm Exp}
\DeclareMathOperator{\Pois}{\mathrm Pois}
\begin{document}
% good reference for equations
\renewcommand{\theequation}{\thesection.\arabic{equation}}
% At least, because ex counter resets at start of section
\renewcommand{\theex}{\thesection.\arabic{ex}}
% better equal by definition
\renewcommand{\eqdef}{\triangleq}

% Transoceanization
\renewcommand{\emptyset}{\varnothing}
\renewcommand{\phi}{\varphi}
\renewcommand{\epsilon}{\varepsilon}

% Львовский называл это dirty tricks,
% но идеалогически единицей курса является именно лекция
\renewcommand{\chaptername}{Лекция}

\newcommand{\Cx}{\mathbb{C}}
\newcommand{\Hx}{\mathbb{H}}
\newcommand{\Zm}[1]{\mathbb{Z}_{#1}}
\newcommand{\fA}{~\forall\;}
\newcommand{\Ex}{~\exists\;}
\newcommand{\Exo}{~\exists\,!\;}
\newcommand{\Pbb}{\mathbb{P}}
\newcommand{\upto}{\nearrow}
% from Львовский with love
\newcommand*{\diff}[1]{\, d#1}


\dmvntitle{Булинский А.В.}{Случайные процессы}{6 семестр, втрой поток}{}{\today}
%FIXME: переносы формул
\include{misc} % TOC и Предметный указатель
%FIXME: заменить | на \mid
\chapter{Случайные блуждания}

\section{Понятие случайного блуждания}

\begin{df}\index{Измеримое!пространство}
	Пусть $V$ \td множество, а $\As$ \td $\si$-алгебра его подмножеств.
	Тогда $(V, \As)$ называется \textit{измеримым пространством}.
\end{df}

\begin{df}\index{Измеримое!отображение}
 	Пусть есть $(V, \As)$ и $(S, \Bs)$ \td два измеримых пространства, 
	$f \cln V \to S$ \td отображение.
	$f$ называется \textit{$\As \divs \Bs$\hизмеримым}, если $\fA B \in \Bs \; f^{-1}(B) \in \As$.
\end{df}
\begin{denote}
	$f\in \As \divs \Bs$.
\end{denote}

\begin{df}\index{Случайный!элемент}
	Пусть есть $(\Om, \Fs, \Pb)$ \td вероятностное пространство,
	$(S, \Bs)$ \td измеримое пространство,
	$Y \cln \Om \to S$ \td отображение.
	Если $Y \in \Fs \divs \Bs$, то $Y$ называется \textit{случайным элементом}.
\end{df}

\begin{ex}
	$S = \R^m$, $\Bc = \Bc(\R)$ \td борелевские множества.
	Тогда при $m > 1$ случайный элемент $Y$ \td случайный вектор;
	если $m = 1$, то $Y$ \td случайная величина.
	$\Pb_Y(B) = \Pb[Y^{-1}(B)]$ \td мера на $\Bc$.

	Легко видеть, что
	$$
		\Pb_Y(B) = \Pb\hc{\om \in \Om | Y(\om) \in B}
	$$
\end{ex}

\begin{df}\index{Распределение!случайного элемента}
 	Пусть $(\Om, \Fs, \Pb)$ \td вероятностное пространство,
	$(S, \Bs)$ \td измеримое пространство,
	$Y \cln \Om \to S$ \td случайный элемент.
	\textit{Распределение вероятностей, индуцированное случайным элементом $Y$,}
	\td это функция на множествах из $\Bs$, задаваемая равенством
	$$
		\Pb_Y (B)\deq \Pb(Y^{-1}(B)), \quad B\in\mathscr{B}.
	$$
\end{df}

\begin{df}\index{Случайный!процесс}
	Пусть $(S_t, \Bs_t)_{t \in T}$ \td семейство измеримых пространств.
	\textit{Случайный процесс, ассоциированный с этим семейством,} \td это семейство случайных элементов
	$X = \hc{X(t) \mid t \in T}$, где $X(t) \cln \Om \to S_t$,
	$X(t) \in \Fs \divs \Bs_t \; \fA t \in T$.
	Здесь $T$ \td это произвольное параметрическое множество,
	$(S_t, \Bs_t)$ \td произвольные измеримые пространства.
\end{df}

\begin{note}
	Если $T \subs \R$, то $t \in T$ интерпретируется как время.
	Если $T = \R$, то время \textit{непрерывно};
	если $T = \Z$ или $T = \Z_+$, то время \textit{дискретно};
	если $T \subs \R^d$, то говорят о \textit{случайном поле}.
\end{note}

\begin{df}
	Случайные элементы $X_1 \sco X_n$ называются \textit{независимыми}, если
	$$
		\Pb \hr{\capkun \hc{X_k \in B_k}} = \prodl{k=1}{n} \Pb(X_k \in B_k),
			\quad \fA B_1 \in \Bs_1 \sco B_n \in \Bs_n.
	$$
\end{df}

\begin{theorem}[Ломницкого-Улама]\index{Теорема!Ломницкого-Улама}
 	Пусть $(S_t, \Bs_t, \Q_t)_{t \in	T}$ \td семейство вероятностных пространств.
	Тогда на некотором $(\Om, \Fs, \Pb)$ существует семейство \textit{независимых} случайных элементов
	$X_t \cln \Om \to S_t$, $X_t \in \Fs \divs \Bs_t$ таких, что $\Pb_{X_t} = \Q_{t}$, $t \in T$.
\end{theorem}

\begin{note}
	Это значит, что на некотором вероятностном пространстве можно
	задать независимое семейство случайных элементов с наперед указанными распределениеми.
	При этом $T$ по-прежнему любое, как и $(S_t, \Bs_t, \Q)_{t \in T}$ \td произвольные вероятностные пространства.
	Независимость здесь означает независимость в совокупности для любого конечного поднабора.
\end{note}

\section{Случайные блуждания}

\begin{df}\index{Случайное блуждание}
	Пусть $X, X_1, X_2\etc$ \td независимые одинаково распределенные случайные векторы	со значениями в $\R^d$.
	\textit{Случайным блужданием в $\R^d$} называется случайный процесс с дискретным временем
	$S = \hc{ S_n, n \ge 0}, n \in \Z_+$ такой, что
	\begin{align*}
			S_0 & \deq x \in \R^d \quad\text{(начальная точка)};\\
			S_n & \deq x + X_1 \spl X_n, \quad n \in \N.
	\end{align*}
\end{df}

\begin{df}\index{Случайное блуждание!простое}
	\textit{Простое случайное блуждание в $\Z^d$} \td это такое случайное блуждание, что
	$$
		\Pb(X = e_k) = \Pb(X = -e_k) = \frac1{2d},
	$$
	где $e_k = (0 \sco 0, \ub{1}_k, 0 \sco 0)$, $k = 1 \sco d$.
\end{df}

\begin{df}\index{Случайное блуждание!простое!возвратное}
	Введем $N \deq \sumnzi \Ibb \hc{S_n = 0} \le \bes$.
	Это, по сути, число попаданий нашего процесса в точку $0$.
	Простое случайное блуждание $S = \hc{S_n, n \ge 0}$ называется \textit{возвратным},
	если $\Pb(N = \bes) = 1$; \textit{невозвратным}, если $\Pb(N < \bes) = 1$.
\end{df}

\begin{note}
	Следует понимать, что хотя определение подразумевает, что $\Pb(N = \bes)$ равно либо 0, либо 1,
	пока что это является недоказанным фактом.
	Это свойство будет следовать из следующей леммы.
\end{note}
\begin{df}
	Число $\tau \deq \inf\hc{ n \in \N : S_{n} = 0}$ ($\tau \deq \bes$,
	если $S_{n} \ne 0 \; \forall\, n \in N$) называется \textit{моментом первого возвращения в 0}.
\end{df}

\begin{lemma}
	Для $ \fA n \in \N \; \Pb(N = n) = \Pb(\tau = \bes)\Pb(\tau < \bes)^{n-1}$.
		\footnote{\textit{от наборщика:} Судя по всему, в лемме подразумевается,
			что начальная точка нашего случайного блуждания \td это 0.}
\end{lemma}

\begin{proof}
	При $n = 1$ формула верна: $\hc{ N = 1} = \hc{ \tau = \bes}$.
Докажем по индукции.

	\begin{gather*}
		\Pb(N = n+1, \tau < \bes) =
		\sumkun \Pb(N = n+1, \tau = k) 
	=	\sumkun \Pb\hr{ \sum_{m=0}^{\bes} \Ibb \hc{ S_{m+k} - S_k = 0 } = n, \tau = k} = \\
	=	\sumkun \Pb \hr{ \sum_{m=0}^{\bes} \Ibb \hc{ S_m = 0} = n} \Pb(\tau = k) 
	=	\sumkun \Pb(N^{\prime} = n)\Pb(\tau = k),
	\end{gather*}
	где $N^{\prime}$ определяется по последовательности
	$X_1^{\prime} = X_{k+1}, X_{2}^{\prime} = X_{k+2}$ и так далее.
	Из того, что $X_i$ \td независиые одинаково распределенные случайные векторы,
	следует, что $N^{\prime}$ и $N$ распределены одинаково.
	Таким образом, получаем, что
	$$
		\Pb(N = n+1, \tau < \bes) = \Pb(N = n)\Pb(\tau < \bes).
	$$
	Заметим теперь, что
	$$
		\Pb(N = n+1) = \Pb(N = n+1, \tau < \bes) + \Pb(N = n+1, \tau = \bes),
	$$
	где второе слагаемое обнуляется из-за того, что $n+1 \ge 2$.
	Из этого следует, что
	$$
		\Pb(N = n+1) = \Pb(N = n)\Pb(\tau < \bes).
	$$
		Пользуемся предположением индукции и получаем, что
	$$
		\Pb(N = n+1) = \Pb(\tau = \bes)\Pb(\tau < \bes)^n,
	$$
	что и завершает доказательство леммы.
\end{proof}

\begin{imp}
	$\Pb(N = \bes)$ равно 0 или 1.
	$\Pb(N < \bes) = 1 \Lra \Pb(\tau < \bes) < 1$.
\end{imp}

\begin{proof}
	Пусть $\Pb(\tau < \bes) < 1$.
	Тогда
	$$
		\Pb(N < \bes) = \sumnui \Pb(N = n) = \sumnui \Pb(\tau = \bes) \Pb(\tau < \bes)^{n-1}
		= \frac{\Pb(\tau = \bes)}{1 - \Pb(\tau < \bes)} = \frac{\Pb(\tau = \bes)}{\Pb(\tau = \bes)} = 1.
	$$
	Это доказывает первое утверждение следствия и импликацию справа налево в формулировке следствия.
	Докажем импликацию слева направо.
	$$
		\Pb(\tau < \bes) = 1 \Ra \Pb \hr{\tau = \bes } = 0 \Ra \Pb(N = n) = 0
			\quad \fA n \in \N \Ra \Pb(N < \bes) = 0.
	$$
	Следствие доказано.
\end{proof}

\begin{theorem}
	Простое случайное блуждание в $\Z^d$ возвратно $\Lra$ $\Ef N = \bes$
	(соответственно, невозвратно $\Lra$ $\Ef N < \bes$).
\end{theorem}

\begin{proof}
	Если $\Ef N < \bes$, то $\Pb(N<\bes) = 1$.
	Пусть теперь $\Pb(N<\bes) = 1$.
	Это равносильно тому, что $\Pb(\tau < \bes) < 1$.
	$$
		\Ef N = \sumnui n\Pb(N=n)
	=	\sumnui n\Pb(\tau = \bes)\Pb(\tau < \bes)^{n-1}
	=	\Pb(\tau = \bes)\sumnui n\Pb(\tau < \bes)^{n-1}.
	$$
	Заметим, что
	$$
		\sumnui np^{n-1} = \hr{\sumnui p^n}^{\prime} = (\frac1{1-p})^{\prime} = \frac1{(1-p)^2}.
	$$
	Тогда, продолжая цепочку равенств, получаем, что
	$$
		\Pb(\tau = \bes)\sumnui n\Pb(\tau < \bes)^{n-1} =
		\frac{\Pb(\tau = \bes)}{(1 - \Pb(\tau < \bes))^2} = \frac1{1 - \Pb(\tau < \bes)},
	$$
	что завершает доказательство теоремы.
\end{proof}

\begin{note}
	Заметим, что поскольку $N = \sumnui \Ibb \hc{S_n = 0}$, то
	$$
		\Ef N = \sumnzi \Ef \Ibb \hc{S_n = 0} = \sumnzi \Pb(S_n = 0),
	$$
	где перестановка местами знаков матожидания и суммы возможна в силу неотрицательности членов ряда.
	Таким образом,
	\begin{center}
		S возвратно $\Lra$ $\sumnzi \Pb(S_n = 0) = \bes$.
	\end{center}
\end{note}

\begin{imp}
	$S$ возвратно при $d = 1$ и $d = 2$.
\end{imp}

\begin{proof}
	$$
		\Pb(S_{2n} = 0) =
		(\frac1{2d})^{2n} \sum_{\substack{n_1 \sco n_d \ge 0 \\ n_1 \spl n_d = n}} \frac{(2n)!}{(n_1!)^2 \ldots (n_d!)^2}.
	$$
	\textit{Случай d = 1}: $\Pb(S_{2n} = 0) = \frac{(2n)!}{(n!)^{2}}(\frac{1}{2})^{2n}$.
	Согласно формуле Стирлинга,
	$$
		m! \sim \hr{\frac m e}^m \sqrt{2 \pi m}, \quad m \to \bes.
	$$
	Соответственно,
	$$
		\Pb(S_{2n} = 0) \sim \frac{1}{\sqrt{\pi n}} \Ra
	$$
	ряд $\sumnzi \frac 1{\sqrt{\pi n}} = \bes \Ra$ блуждание возвратно.
	Аналогично рассматривается \textit{случай d = 2}:
	$$
		\Pb(S_{2n} = 0) = \ldots = \hc{ \frac{(2n)!}{(n!)^2}(\frac 1{2})^{2n} }^{2} \sim \frac{1}{\pi n} \Ra
	$$
	ряд тоже разойдется $\Ra$ блуждание возвратно.
	Теорема доказана.
\end{proof}

\section{Исследование случайного блуждания с помощью характеристической функции}

\begin{theorem}
	Для простого случайного блуждания в $\Z^{d}$
	$$
		\Ef N = \lim_{c \up 1} \frac 1{(2 \pi)^d} \ints{[-\pi, \pi]^d} \frac1{1 - c \phi (t)}\diff{t},
	$$
	где $\phi (t)$ \td характеристическая функция X, $t \in \mathbb{R}^{d}$.
\end{theorem}

\begin{proof}
	$$
		\ints{[-\pi, \pi]} \frac{e^{inx}}{2 \pi} \diff{x} = \bcase{
				1, && n &= 0, \\
				0, && n &\ne 0.
		}
	$$
	Следовательно,
	$$
		\Ibb \hc{S_n = 0}
	=	\prod_{k=1}^d \Ibb \hc{ S_n^{(k)} = 0}
	=	\prod_{k=1}^d \ints{[-\pi, \pi]} \frac{e^{i S_n^{(k)} t_k}}{2 \pi}\diff{ t_k}
	=	\frac1{(2 \pi)^d} \ints{[-\pi, \pi]^d} e^{i (S_n, t)}\diff{ t}.
	$$
	По теореме Фубини
	$$
		\Ef \Ibb \hc{S_n = 0}
	=	\Ef \frac1{(2 \pi)^d} \ints{[-\pi, \pi]^d} e^{i (S_n, t)}\diff{ t}
	=	\frac1{(2 \pi)^d} \ints{[-\pi, \pi]^d} \Ef e^{i (S_n, t)}\diff{ t}.
	$$
	Заметим, что
	$$
		\Ef e^{i (S_n, t)} = \prod_{k=1}^n \phi_{X_k} (t) = (\phi (t))^n.
	$$
	Тогда
	$$
		\Ef \Ibb (S_n = 0) = \Pb(S_n = 0)
	=	\frac1{(2 \pi)^d}\ints{[-\pi, \pi]^d} \hr{\phi (t)}^n\diff{ t}.
	$$
	Из этого следует, что
	$$
		\sumnzi c^n \Pb(S_n = 0)
	=	\frac1{(2 \pi)^d} \ints{[-\pi, \pi]^d} \sumnzi (c \phi(t))^n\diff{t},\quad\text{где $0 < c < 1$}.
	$$
	Поскольку $|c \phi| \le c < 1$, то
	$$
		\frac1{(2 \pi)^d} \ints{[-\pi, \pi]^d} \sumnzi (c \phi(t))^n\diff{ t}
	=	\frac1{(2 \pi)^d} \ints{[-\pi, \pi]^d} \frac1{1 - c \phi (t)}\diff{ t}
	$$
	по формуле для суммы бесконечно убывающей геометрической прогрессии.
	Осталось только заметить, что
	$$
		\sumnzi c^n \Pb(S_n = 0) \to \sumnzi \Pb(S_n = 0) = \Ef N, \quad c \uparrow 1,
	$$
	что и завершает доказательство теоремы.
\end{proof}

\begin{imp}
	При $d \ge 3$ простое случайное блуждание невозвратно.
\end{imp}

\begin{note}
	Можно говорить и о случайных блужданиях в $\R^d$, если $X_i: \Om \to \R^d$.
	Но тогда о возвратности приходится говорить в терминах
	бесконечно частого попадания в $\epsilon$\hокрестность точки $x$.
\end{note}

\begin{df}\index{Множество!возвратности}
	Пусть есть случайное блуждание $S$ на $\R^d$.
	Тогда \textit{множество возвратности} случайного блуждания $S$ \td это множество
	$$
		R(S) = \hc{ x \in \R^d : \text{блуждание возвратно в окрестности точки } x}.
	$$
\end{df}

\begin{df}\index{Множество!достижимости}
	Пусть есть случайное блуждание $S$ на $\R^d$.
	Тогда \textit{точки, достижимые случайным блужданием $S$,} \td это множество $P(S)$ такое, что
	$$
		\fA z \in P(S) \; \fA \epsilon > 0 \; \Ex n : \; \Pb( | S_{n} - z | < \epsilon) > 0.
	$$
\end{df}

\begin{theorem}[Чжуна-Фукса]\index{Теорема!Чжуна-Фукса}
	Если $R(S) \neq \emptyset$, то $R(S) = P(S)$.
\end{theorem}

\begin{imp}
	Если $0 \in R(S)$, то $R(S) = P(S)$;
	%TODO Львовский
	если $0 \notin R(S)$, то	$R(S) = \emptyset$.
\end{imp}

\chapter{Ветвящиеся процессы и процессы восстановления}

\section{Модель Гальтона\DВатсона}\index{Модель Гальтона-Ватсона}

\paragraph{Описание модели.}

Пусть $\hc{ \xi, \xi_{n, k} \mid n, k \in \N}$ \td массив независимых одинаково распределенных случайных величин,
\[
	\Pf (\xi = m) = p_m \ge 0, \; m \in \Z_{+} = \hc{0, 1, 2\etc}.
\]
Такие существуют в силу теоремы Ломницкого--Улама.
Положим
\begin{align*}
	Z_0(\omega) &\deq 1,\\
	Z_n(\omega) &\deq \sum_{k = 1}^{Z_{n - 1}(\omega)} \xi_{n, k}(\omega) \quad \text{для $n \in \N$}.
\end{align*}
Здесь подразумевается, что если $Z_{n-1}(\omega) = 0$, то и вся сумма равна нулю.
Таким образом, рассматривается сумма случайного числа случайных величин.
Определим
$A = \hc{\omega \mid \Ex n = n(\omega): Z_{n}(\omega) = 0}$ \td \index{Вырождение}\textit{событие вырождения популяции}.
Заметим, что если $Z_n(\omega) = 0$, то $Z_{n+1}(\omega) = 0$.
Таким образом,
$\hc{ Z_n = 0} \subs \hc{ Z_{n+1} = 0}$ и $A = \cupnui \hc{ Z_{n} = 0}$.

По свойству непрерывности вероятностной меры,
\[
	\Pf(A) = \lim_{n \to \bes} \Pf(Z_n = 0).
\]

\begin{df}\index{Производящая функция}
	Пусть дана последовательность $(a_n)_{n=0}^{\bes}$ неотрицательных чисел такая,
	что $\sumnzi a_n = 1$.
	\textit{Производящая функция} для этой последовательности \td это
	\[
		f(s) \deq \sumkzi s^k a_k, \quad |s| \le 1
	\]
	(нас в основном будут интересовать $s \in [0, 1]$).
\end{df}

Заметим, что если $a_k = \Pf(Y = k), k = 0, 1,\; \ldots$ , то
\[
	f_Y(s) = \sumkzi s^k \Pf(Y = k) = \Ef s^Y, \quad s \in [0, 1].
\]

\begin{lemma}
	\label{lem1}
	Вероятность $\Pf(A)$ является корнем уравнения $\psi(p) = p$, где $\psi = f_{\xi}$ и $p \in [0, 1]$.
\end{lemma}

\begin{proof}
	\begin{multline*}
		f_{Z_n}(s)
	=	\Ef s^{Z_n}
	=	\Ef \hr{s^{\sum_{k = 1}^{Z_{n - 1}} \xi_{n, k}}} = \\
	=	\sumjzi \Ef \hs{ \hr{ s^{\sum_{k = 1}^{Z_{n - 1}} \xi_{n, k}} } \Ibb \hc{ Z_{n - 1} = j}} = \\
	=	\sumjzi \Ef \hs{ \hr{ s^{\sum_{k = 1}^j \xi_{n, k}} } \Ibb \hc{ Z_{n - 1} = j} }.
	\end{multline*}
	Поскольку $\sigma \hc{ Z_r} \subs \sigma \hc{ \xi_{m, k}, \; m = 1 \sco r,\; k \in \N}$,
	которая независима с $\sigma \hc{\xi_{n, k}, \; k \in \N}$
	(строгое и полное обоснование остается в качестве упражнения), то
	\begin{multline*}
		\sumjzi \Ef \hs{ \hc{ s^{\sum_{k = 1}^j \xi_{n, k}} } \Ibb \hc{ Z_{n-1} = j } }
	=	\sumjzi \Ef \hr{s^{\sum_{k = 1}^j \xi_{n, k}}} \Ef \Ibb \hc{ Z_{n - 1} = j} = \\
	=	\sumjzi \Ef \hr{s^{\sum_{k=1}^j \xi_{n, k}}} \Pf ( Z_{n - 1} = j )
	=	\sumjzi \prod_{k = 1}^j \Ef s^{\xi_{n, k}} \Pf (Z_{n - 1} = j) = \\
	=	\sumjzi \psi_{\xi}^j (s) \Pf (Z_{n - 1} = j)
	=	f_{Z_{n - 1}} \hr{\psi_{\xi} (s)}
	\end{multline*}
	в силу независимости и одинаковой распределенности $\xi_{n, k}$ и определения производящей функции.
	Таким образом,
	\[
		f_{Z_n} (s) = f_{Z_{n - 1}} \hr{ \psi_{\xi} (s)}, \quad s \in [0, 1]{.}
	\]
	Подставим $s = 0$ и получим, что
	\[
		f_{Z_n} (0) = f_{Z_{n - 1}} \hr{ \psi_{\xi} (0)}
	\]
	Заметим, что
	\[
		f_{Z_n}(s)
	=	f_{Z_{n - 1}}(\psi_{\xi}(s))
	=	f_{Z_{n - 2}} \hr{\psi_{\xi} \hr{ \psi_{\xi} (s) } }
	=	\ldots
	=	\ub{\psi_{\xi} (\psi_{\xi} \ldots (\psi_{\xi}(s)) \ldots ) }_{\text{$n$ итераций}}
	=	\psi_{\xi} (f_{Z_{n - 1}} (s)).
	\]
	Тогда при $s = 0$ имеем, что
	\[
		\Pf (Z_n = 0) = \psi_{\xi} \hr{ \Pf\hr{Z_{n - 1} = 0}}.
	\]
	Но $\Pf(Z_n = 0) \upto \Pf(A)$ при $n \to \bes$ и $\psi_{\xi}$ непрерывна на $[0, 1]$.
	Переходим к пределу при $n \to \bes$.
	Тогда
	\[
		\Pf(A) = \psi_{\xi} (\Pf(A)),
	\]
	то есть $\Pf(A)$ \td корень уравнения $p = \psi_{\xi}(p)$, $p \in [0, 1]$.
\end{proof}

\begin{theorem}
	Вероятность $p$ вырождения процесса Гальтона\DВатсона есть \textbf{наименьший} корень уравнения
	\begin{equation}
		\label{eq1}
		\psi(p) = p, \quad p \in [0, 1],
	\end{equation}
	где $\psi = \psi_{\xi}$.
\end{theorem}

\begin{proof}
	Пусть $p_0 \deq \Pf(\xi = 0) = 0$.
	Тогда
	\[
		\Pf(\xi \ge 1) = 1, \quad \Pf\hr{\caps{n,k} \hc{ \xi_{n,k} \ge 1} } = 1.
	\]
	Поэтому $Z_n \ge 1$ при $\fA n$, то есть $\Pf(A)$ \td наименьший корень уравнения~\eqref{eq1}.

	Пусть теперь $p_0 = 1$.
	Тогда $\Pf(\xi = 0)=1 \Ra \Pf(A)$ \td наименьший корень уравнения~\eqref{eq1}.

	Пусть, наконец, $0 < p_{0} < 1$.
	Из этого следует, что $\Ex m\in\N:\; p_m > 0$, а значит, $\psi$ строго возрастает на $[0, 1]$.
	Рассмотрим
	\[
		\Delta_n = \hsr{\psi_{n}(0), \psi_{n + 1}(0)},\quad n = 0, 1, 2 \etc ,
	\]
	где $\psi_n(s)$ \td это производящая функция $Z_n$.

	Пусть $s \in \Delta_n$.
	Тогда из монотонности $\psi$ на $[0, 1]$ получаем, что
	\[
		\psi(s) - s > \psi(\psi_n(0)) - \psi_{n + 1}(0)  =  \psi_{n + 1}(0) - \psi_{n + 1}(0) = 0,
	\]
	что означает, что у уравнения~\eqref{eq1} нет корней на $\Delta_n \; \fA n \in \Z_{+}$.
	Заметим, что
	\[
		\cupnzi \Delta_n = \hsr{0, \Pf(A)}, \quad \psi_n(0) \upto \Pf(A).
	\]
	По лемме \ref{lem1} $\Pf(A)$ является корнем уравнения \eqref{eq1}.
	Следовательно, показано, что $\Pf(A)$ \td наименьший корень, что и требовалось доказать.
\end{proof}

\begin{theorem}
	\begin{points}{0}
		\item\label{firth} Вероятность вырождения $\Pf(A)$ есть нуль $\Lra$ $p_{0} = 0$.
		\item\label{secth} Пусть $p_{0} > 0$.
			Тогда при $\Ef \xi \le 1$ имеем $\Pf (A) = 1$,
			при $\Ef \xi > 1$ имеем $\Pf(A) < 1$.
	\end{points}
\end{theorem}

\begin{proof}
	Докажем \autoref{firth}.
	Пусть $\Pf(A) = 0$.
	Тогда $p_{0} = 0$, потому что иначе была бы ненулевая вероятность вымирания $\Pf(A) > \Pf(Z_1 = 0) = p_0$.
	В другую сторону, если $p_0 = 0$, то вымирания не происходит (почти наверное) из-за того,
	что у каждой частицы есть как минимум один потомок (почти наверное).

	Докажем \autoref{secth}.
	Пусть $\mu = \Ef\xi \le 1$.
	Покажем, что в таком случае у уравнения \eqref{eq1} будет единственный корень, равный $1$.
	\[
		\psi_{\xi}^{\prime} (z)
	=	\sumkui kz^{k - 1}\Pf(\xi = k) \Ra \psi_{\xi}^{\prime} (z) > 0, \quad \text{ при $z > 0$,}
	\]
	если только $\xi$ не тождественно равна нулю (в противном случае утверждение теоремы выполнено).
	Заметим также, что $\psi_{\xi}^{\prime} (z)$ возрастает на $z > 0$.
	Воспользуемся формулой Лагранжа:
	\[
		1 - \psi_{\xi} (z)
	=	\psi_{\xi} (1) - \psi_{\xi} (z)
	=	\psi_{\xi}^{\prime} (\theta) (1 - z)
	<	\psi_{\xi}^{\prime} (1) (1-z) \le 1 - z,
	\]
	где $z \in (0, 1)$, в силу монотонности $\psi_{\xi}^{\prime} (z)$.
	Следовательно, если $z < 1$, то
	\[
		1 - \psi_{\xi}(z) < 1 - z,
	\]
	то есть $z = 1$ \td это единственный корень уравнения \eqref{eq1}.
	Значит, $P(A) = 1$.

	Пусть $\mu = \Ef\xi > 1$.
	Покажем, что в таком случае у уравнения \eqref{eq1} есть два корня, один из которых строго меньше единицы.
	\[
		\psi_{\xi}^{\prime\prime}(z)
	=	\sum\limits_{k=2}^{\bes} k (k - 1) z^{k - 2} \Pf(\xi = k),
	\]
	следовательно, $\psi_{\xi}^{\prime\prime}(z)$ монотонно возрастает и больше нуля при $z > 0$.
	Из этого следует, что $1 - \psi_{\xi}^{\prime} (z)$ строго убывает, причем
	\begin{align*}
		& 1 - \psi_{\xi}^{\prime} (0) = 1 - \Pf(\xi = 1) > 0 , \\
		& 1 - \psi_{\xi}^{\prime} (1) = 1 - \mu < 0 .
	\end{align*}
	Рассмотрим теперь $z - \psi_{\xi} (z)$ при $z = 0$.
	Поскольку $1 - \psi_{\xi} (1) = 0$, производная этой функции монотонно убывает,
	а $0 - \psi_{\xi} (0) = -\Pf(\xi = 0) < 0$,
	то график функции $z - \psi_{\xi} (z)$ пересечет ось абсцисс в двух точках,
	одна из которых будет лежать в интервале $(0, 1)$.
	Так как вероятность вырождения $\Pf(A)$ равна наименьшему корню уравнения \eqref{eq1},
	то $\Pf(A) < 1$, что и требовалось доказать.
\end{proof}

\begin{imp}
	Пусть $\Ef \xi < \bes$.
Тогда $\Ef Z_{n} = (\Ef \xi)^{n},\; n \in \N{.}$
\end{imp}

\begin{proof}
	Доказательство проводится по индукции.

	База индукции: $n=1 \Ra \Ef Z_{1} = \Ef \xi$.

	Индуктивный переход:
	\[
		\Ef Z_{n}
	=	\Ef \hr{ \sum_{k = 1}^{Z_{n - 1}} \xi_{n,k} }
	=	\sumjzi j \Ef \xi \Pf(Z_{n - 1} = j)
	=	\Ef \xi \Ef Z_{n - 1}
	=	\hr{\Ef \xi}^n.
	\]
\end{proof}

\begin{df}

	При $\Ef \xi < 1$ процесс называется \textit{докритическим.}

	При $\Ef \xi = 1$ процесс называется \textit{критическим.}

	При $\Ef \xi > 1$ процесс называется \textit{надкритическим.}
\end{df}

\section{Процессы восстановления}

\begin{df}\index{Процесс!восстановления}
	Пусть $S_n = X_1 \spl X_n$, где $n \in \N$, 
	$X, X_1, X_{2} \etc$ \td независимые одинаково распределенные случайные величины, $X \ge 0$.
	Положим
	\begin{align*}
		Z(0) &\deq 0;\\
		Z(t) &\deq \sup \hc{ n \in \N \mid S_n \le t }, \quad t > 0.
	\end{align*}
	(здесь считаем, что $\sup \varnothing \deq \bes$).
	Таким образом,
	\[
		Z(t, \omega) = \sup \hc{ n \in \N \mid S_n(\omega) \le t }.
	\]
	Иными словами,
	\[
		\hc{ Z(t) \ge n} = \hc{ S_n \le t}.
	\]
	Так определенный процесс $Z(t)$ называется \textit{процессом восстановления}.
\end{df}

\begin{note}
	Полезно заметить, что
	\[
		Z(t) = \sumnui \Ibb \hc{ S_n \le t},\quad t > 0.
	\]
\end{note}
\begin{df}
%FIXME
\phantomsection
\label{dfstar}
	Рассмотрим \textit{процесс восстановления} $\hc{ Z^\star(t),\; t \ge 0 }$,
	который строится по $Y, Y_{1}, Y_{2}\etc$ \td независимым одинаково распределенным случайным величинам,
	где $\Pf(Y = \alpha) = p \in (0, 1)$, $\Pf(Y = 0) = 1 - p$.
	Исключаем из рассмотрения случай, когда $Y = C = \const$:
		если $C = 0$, то $Z(t) = \bes \; \fA t > 0$;
		если же $C > 0$, то $Z(t) = \hs{\frac t c}$.
\end{df}

\begin{lemma}
	Для $l = 0, 1, 2\etc$
	\[
		\Pf(Z^{\star}(t) = m) =
		%FIXME?
		\bcase{
			C_m^j p^{\hs{\frac t \alpha} + 1} q^{m - \hs{\frac t \alpha}}, &\quad\text{если  $m \ge j$;} \\
			0,&\quad\text{если $m < j$.}
		}
	\]
\end{lemma}

\begin{thebibliography}{20}
\bibitem{B-Sh} Булинский, Ширяев,
		Теория случайных процессов
\end{thebibliography}

\end{document}
