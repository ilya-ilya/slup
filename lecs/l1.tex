\chapter{Случайные блуждания}

\section{Понятие случайного блуждания}

\begin{df}\index{Измеримое!пространство}
	Пусть $V$ \td множество, а $\As$ \td $\si$-алгебра его подмножеств.
	Тогда $(V, \As)$ называется \textit{измеримым пространством}.
\end{df}

\begin{df}\index{Измеримое!отображение}
 	Пусть есть $(V, \As)$ и $(S, \Bs)$ \td два измеримых пространства, 
	$f \cln V \to S$ \td отображение.
	$f$ называется \textit{$\As \divs \Bs$\hизмеримым}, если $\fA B \in \Bs \; f^{-1}(B) \in \As$.
\end{df}
\begin{denote}
	$f\in \As \divs \Bs$.
\end{denote}

\begin{df}\index{Случайный!элемент}
	Пусть есть $(\Om, \Fs, \Pf)$ \td вероятностное пространство,
	$(S, \Bs)$ \td измеримое пространство,
	$Y \cln \Om \to S$ \td отображение.
	Если $Y \in \Fs \divs \Bs$, то $Y$ называется \textit{случайным элементом}.
\end{df}

\begin{ex}
	$S = \R^m$, $\Bc = \Bc(\R)$ \td борелевские множества.
	Тогда при $m > 1$ случайный элемент $Y$ \td случайный вектор;
	если $m = 1$, то $Y$ \td случайная величина.
	$\Pf_Y(B) = \Pf[Y^{-1}(B)]$ \td мера на $\Bc$.

	Легко видеть, что
	\[
		\Pf_Y(B) = \Pf\hc{\om \in \Om \mid Y(\om) \in B}
	\]
\end{ex}

\begin{df}\index{Распределение!случайного элемента}
 	Пусть $(\Om, \Fs, \Pf)$ \td вероятностное пространство,
	$(S, \Bs)$ \td измеримое пространство,
	$Y \cln \Om \to S$ \td случайный элемент.
	\textit{Распределение вероятностей, индуцированное случайным элементом $Y$,}
	\td это функция на множествах из $\Bs$, задаваемая равенством
	\[
		\Pf_Y (B)\deq \Pf(Y^{-1}(B)), \quad B\in\mathscr{B}.
	\]
\end{df}

\begin{df}\index{Случайный!процесс}
	Пусть $(S_t, \Bs_t)_{t \in T}$ \td семейство измеримых пространств.
	\textit{Случайный процесс, ассоциированный с этим семейством,} \td это семейство случайных элементов
	$X = \hc{X(t) \mid t \in T}$, где $X(t) \cln \Om \to S_t$,
	$X(t) \in \Fs \divs \Bs_t \; \fA t \in T$.
	Здесь $T$ \td это произвольное параметрическое множество,
	$(S_t, \Bs_t)$ \td произвольные измеримые пространства.
\end{df}

\begin{note}
	Если $T \subs \R$, то $t \in T$ интерпретируется как время.
	Если $T = \R$, то время \textit{непрерывно};
	если $T = \Z$ или $T = \Z_+$, то время \textit{дискретно};
	если $T \subs \R^d$, то говорят о \textit{случайном поле}.
\end{note}

\begin{df}
	Случайные элементы $X_1 \sco X_n$ называются \textit{независимыми}, если
	\[
		\Pf \hr{\capkun \hc{X_k \in B_k}} = \prodl{k=1}{n} \Pf(X_k \in B_k),
			\quad \fA B_1 \in \Bs_1 \sco B_n \in \Bs_n.
	\]
\end{df}

\begin{theorem}[Ломницкого-Улама]\index{Теорема!Ломницкого-Улама}
 	Пусть $(S_t, \Bs_t, \Q_t)_{t \in	T}$ \td семейство вероятностных пространств.
	Тогда на некотором $(\Om, \Fs, \Pf)$ существует семейство \textit{независимых} случайных элементов
	$X_t \cln \Om \to S_t$, $X_t \in \Fs \divs \Bs_t$ таких, что $\Pf_{X_t} = \Q_{t}$, $t \in T$.
\end{theorem}

\begin{note}
	Это значит, что на некотором вероятностном пространстве можно
	задать независимое семейство случайных элементов с наперед указанными распределениеми.
	При этом $T$ по-прежнему любое, как и $(S_t, \Bs_t, \Q)_{t \in T}$ \td произвольные вероятностные пространства.
	Независимость здесь означает независимость в совокупности для любого конечного поднабора.
\end{note}

\section{Случайные блуждания}

\begin{df}\index{Случайное блуждание}
	Пусть $X, X_1, X_2\etc$ \td независимые одинаково распределенные случайные векторы	со значениями в $\R^d$.
	\textit{Случайным блужданием в $\R^d$} называется случайный процесс с дискретным временем
	$S = \hc{ S_n, n \ge 0}, n \in \Z_+$ такой, что
	\begin{align*}
			S_0 & \deq x \in \R^d \quad\text{(начальная точка)};\\
			S_n & \deq x + X_1 \spl X_n, \quad n \in \N.
	\end{align*}
\end{df}

\begin{df}\index{Случайное блуждание!простое}
	\textit{Простое случайное блуждание в $\Z^d$} \td это такое случайное блуждание, что
	\[
		\Pf(X = e_k) = \Pf(X = -e_k) = \frac1{2d},
	\]
	где $e_k = (0 \sco 0, \ub{1}_k, 0 \sco 0)$, $k = 1 \sco d$.
\end{df}

\begin{df}\index{Случайное блуждание!простое!возвратное}
	Введем $N \deq \sumnzi \Ibb \hc{S_n = 0} \le \bes$.
	Это, по сути, число попаданий нашего процесса в точку $0$.
	Простое случайное блуждание $S = \hc{S_n, n \ge 0}$ называется \textit{возвратным},
	если $\Pf(N = \bes) = 1$; \textit{невозвратным}, если $\Pf(N < \bes) = 1$.
\end{df}

\begin{note}
	Следует понимать, что хотя определение подразумевает, что $\Pf(N = \bes)$ равно либо 0, либо 1,
	пока что это является недоказанным фактом.
	Это свойство будет следовать из следующей леммы.
\end{note}
\begin{df}
	Число $\tau \deq \inf\hc{ n \in \N : S_{n} = 0}$ ($\tau \deq \bes$,
	если $S_{n} \ne 0 \; \forall\, n \in N$) называется \textit{моментом первого возвращения в 0}.
\end{df}

\begin{lemma}
	Для $ \fA n \in \N \; \Pf(N = n) = \Pf(\tau = \bes)\Pf(\tau < \bes)^{n-1}$.
		\footnote{\textit{от наборщика:} Судя по всему, в лемме подразумевается,
			что начальная точка нашего случайного блуждания \td это 0.}
\end{lemma}

\begin{proof}
	При $n = 1$ формула верна: $\hc{ N = 1} = \hc{ \tau = \bes}$.
Докажем по индукции.

	\begin{gather*}
		\Pf(N = n+1, \tau < \bes) =
		\sumkun \Pf(N = n+1, \tau = k) 
	=	\sumkun \Pf\hr{ \sum_{m=0}^{\bes} \Ibb \hc{ S_{m+k} - S_k = 0 } = n, \tau = k} = \\
	=	\sumkun \Pf \hr{ \sum_{m=0}^{\bes} \Ibb \hc{ S_m = 0} = n} \Pf(\tau = k) 
	=	\sumkun \Pf(N^{\prime} = n)\Pf(\tau = k),
	\end{gather*}
	где $N^{\prime}$ определяется по последовательности
	$X_1^{\prime} = X_{k+1}, X_{2}^{\prime} = X_{k+2}$ и так далее.
	Из того, что $X_i$ \td независиые одинаково распределенные случайные векторы,
	следует, что $N^{\prime}$ и $N$ распределены одинаково.
	Таким образом, получаем, что
	\[
		\Pf(N = n+1, \tau < \bes) = \Pf(N = n)\Pf(\tau < \bes).
	\]
	Заметим теперь, что
	\[
		\Pf(N = n+1) = \Pf(N = n+1, \tau < \bes) + \Pf(N = n+1, \tau = \bes),
	\]
	где второе слагаемое обнуляется из-за того, что $n+1 \ge 2$.
	Из этого следует, что
	\[
		\Pf(N = n+1) = \Pf(N = n)\Pf(\tau < \bes).
	\]
		Пользуемся предположением индукции и получаем, что
	\[
		\Pf(N = n+1) = \Pf(\tau = \bes)\Pf(\tau < \bes)^n,
	\]
	что и завершает доказательство леммы.
\end{proof}

\begin{imp}
	$\Pf(N = \bes)$ равно 0 или 1.
	$\Pf(N < \bes) = 1 \Lra \Pf(\tau < \bes) < 1$.
\end{imp}

\begin{proof}
	Пусть $\Pf(\tau < \bes) < 1$.
	Тогда
	\[
		\Pf(N < \bes) = \sumnui \Pf(N = n) = \sumnui \Pf(\tau = \bes) \Pf(\tau < \bes)^{n-1}
		= \frac{\Pf(\tau = \bes)}{1 - \Pf(\tau < \bes)} = \frac{\Pf(\tau = \bes)}{\Pf(\tau = \bes)} = 1.
	\]
	Это доказывает первое утверждение следствия и импликацию справа налево в формулировке следствия.
	Докажем импликацию слева направо.
	\[
		\Pf(\tau < \bes) = 1 \Ra \Pf \hr{\tau = \bes } = 0 \Ra \Pf(N = n) = 0
			\quad \fA n \in \N \Ra \Pf(N < \bes) = 0.
	\]
	Следствие доказано.
\end{proof}

\begin{theorem}
	Простое случайное блуждание в $\Z^d$ возвратно $\Lra$ $\Ef N = \bes$
	(соответственно, невозвратно $\Lra$ $\Ef N < \bes$).
\end{theorem}

\begin{proof}
	Если $\Ef N < \bes$, то $\Pf(N<\bes) = 1$.
	Пусть теперь $\Pf(N<\bes) = 1$.
	Это равносильно тому, что $\Pf(\tau < \bes) < 1$.
	\[
		\Ef N = \sumnui n\Pf(N=n)
	=	\sumnui n\Pf(\tau = \bes)\Pf(\tau < \bes)^{n-1}
	=	\Pf(\tau = \bes)\sumnui n\Pf(\tau < \bes)^{n-1}.
	\]
	Заметим, что
	\[
		\sumnui np^{n-1} = \hr{\sumnui p^n}^{\prime} = (\frac1{1-p})^{\prime} = \frac1{(1-p)^2}.
	\]
	Тогда, продолжая цепочку равенств, получаем, что
	\[
		\Pf(\tau = \bes)\sumnui n\Pf(\tau < \bes)^{n-1} =
		\frac{\Pf(\tau = \bes)}{(1 - \Pf(\tau < \bes))^2} = \frac1{1 - \Pf(\tau < \bes)},
	\]
	что завершает доказательство теоремы.
\end{proof}

\begin{note}
	Заметим, что поскольку $N = \sumnui \Ibb \hc{S_n = 0}$, то
	\[
		\Ef N = \sumnzi \Ef \Ibb \hc{S_n = 0} = \sumnzi \Pf(S_n = 0),
	\]
	где перестановка местами знаков матожидания и суммы возможна в силу неотрицательности членов ряда.
	Таким образом,
	\begin{center}
		S возвратно $\Lra$ $\sumnzi \Pf(S_n = 0) = \bes$.
	\end{center}
\end{note}

\begin{imp}
	$S$ возвратно при $d = 1$ и $d = 2$.
\end{imp}

\begin{proof}
	\[
		\Pf(S_{2n} = 0) =
		(\frac1{2d})^{2n} \sum_{\substack{n_1 \sco n_d \ge 0 \\ n_1 \spl n_d = n}} \frac{(2n)!}{(n_1!)^2 \ldots (n_d!)^2}.
	\]
	\textit{Случай d = 1}: $\Pf(S_{2n} = 0) = \frac{(2n)!}{(n!)^{2}}(\frac{1}{2})^{2n}$.
	Согласно формуле Стирлинга,
	\[
		m! \sim \hr{\frac m e}^m \sqrt{2 \pi m}, \quad m \to \bes.
	\]
	Соответственно,
	\[
		\Pf(S_{2n} = 0) \sim \frac{1}{\sqrt{\pi n}} \Ra
	\]
	ряд $\sumnzi \frac 1{\sqrt{\pi n}} = \bes \Ra$ блуждание возвратно.
	Аналогично рассматривается \textit{случай d = 2}:
	\[
		\Pf(S_{2n} = 0) = \ldots = \hc{ \frac{(2n)!}{(n!)^2}(\frac 1{2})^{2n} }^{2} \sim \frac{1}{\pi n} \Ra
	\]
	ряд тоже разойдется $\Ra$ блуждание возвратно.
	Теорема доказана.
\end{proof}

\section{Исследование случайного блуждания с помощью характеристической функции}

\begin{theorem}
	Для простого случайного блуждания в $\Z^{d}$
	\[
		\Ef N = \lim_{c \up 1} \frac 1{(2 \pi)^d} \ints{[-\pi, \pi]^d} \frac1{1 - c \phi (t)}\diff{t},
	\]
	где $\phi (t)$ \td характеристическая функция X, $t \in \mathbb{R}^{d}$.
\end{theorem}

\begin{proof}
	\[
		\ints{[-\pi, \pi]} \frac{e^{inx}}{2 \pi} \diff{x} = \bcase{
				1, && n &= 0, \\
				0, && n &\ne 0.
		}
	\]
	Следовательно,
	\[
		\Ibb \hc{S_n = 0}
	=	\prod_{k=1}^d \Ibb \hc{ S_n^{(k)} = 0}
	=	\prod_{k=1}^d \ints{[-\pi, \pi]} \frac{e^{i S_n^{(k)} t_k}}{2 \pi}\diff{ t_k}
	=	\frac1{(2 \pi)^d} \ints{[-\pi, \pi]^d} e^{i (S_n, t)}\diff{ t}.
	\]
	По теореме Фубини
	\[
		\Ef \Ibb \hc{S_n = 0}
	=	\Ef \frac1{(2 \pi)^d} \ints{[-\pi, \pi]^d} e^{i (S_n, t)}\diff{ t}
	=	\frac1{(2 \pi)^d} \ints{[-\pi, \pi]^d} \Ef e^{i (S_n, t)}\diff{ t}.
	\]
	Заметим, что
	\[
		\Ef e^{i (S_n, t)} = \prod_{k=1}^n \phi_{X_k} (t) = (\phi (t))^n.
	\]
	Тогда
	\[
		\Ef \Ibb (S_n = 0) = \Pf(S_n = 0)
	=	\frac1{(2 \pi)^d}\ints{[-\pi, \pi]^d} \hr{\phi (t)}^n\diff{ t}.
	\]
	Из этого следует, что
	\[
		\sumnzi c^n \Pf(S_n = 0)
	=	\frac1{(2 \pi)^d} \ints{[-\pi, \pi]^d} \sumnzi (c \phi(t))^n\diff{t},\quad\text{где $0 < c < 1$}.
	\]
	Поскольку $|c \phi| \le c < 1$, то
	\[
		\frac1{(2 \pi)^d} \ints{[-\pi, \pi]^d} \sumnzi (c \phi(t))^n\diff{ t}
	=	\frac1{(2 \pi)^d} \ints{[-\pi, \pi]^d} \frac1{1 - c \phi (t)}\diff{ t}
	\]
	по формуле для суммы бесконечно убывающей геометрической прогрессии.
	Осталось только заметить, что
	\[
		\sumnzi c^n \Pf(S_n = 0) \to \sumnzi \Pf(S_n = 0) = \Ef N, \quad c \uparrow 1,
	\]
	что и завершает доказательство теоремы.
\end{proof}

\begin{imp}
	При $d \ge 3$ простое случайное блуждание невозвратно.
\end{imp}

\begin{note}
	Можно говорить и о случайных блужданиях в $\R^d$, если $X_i: \Om \to \R^d$.
	Но тогда о возвратности приходится говорить в терминах
	бесконечно частого попадания в $\epsilon$\hокрестность точки $x$.
\end{note}

\begin{df}\index{Множество!возвратности}
	Пусть есть случайное блуждание $S$ на $\R^d$.
	Тогда \textit{множество возвратности} случайного блуждания $S$ \td это множество
	\[
		R(S) = \hc{ x \in \R^d : \text{блуждание возвратно в окрестности точки } x}.
	\]
\end{df}

\begin{df}\index{Множество!достижимости}
	Пусть есть случайное блуждание $S$ на $\R^d$.
	Тогда \textit{точки, достижимые случайным блужданием $S$,} \td это множество $P(S)$ такое, что
	\[
		\fA z \in P(S) \; \fA \epsilon > 0 \; \Ex n : \; \Pf( | S_{n} - z | < \epsilon) > 0.
	\]
\end{df}

\begin{theorem}[Чжуна-Фукса]\index{Теорема!Чжуна-Фукса}
	Если $R(S) \neq \emptyset$, то $R(S) = P(S)$.
\end{theorem}

\begin{imp}
	Если $0 \in R(S)$, то $R(S) = P(S)$;
	если $0 \notin R(S)$, то	$R(S) = \emptyset$.
\end{imp}
